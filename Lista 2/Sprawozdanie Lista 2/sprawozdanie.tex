\documentclass{article}
\usepackage[utf8]{inputenc}
\usepackage{fancyhdr}
\usepackage{polski}
\usepackage{mathtools}
\usepackage{ulem}
\usepackage[margin = 1cm]{geometry}
\usepackage{hhline}
\usepackage{array}
\usepackage{booktabs}
\usepackage{tabularx}
\usepackage{colortbl}
\usepackage{longtable}
\usepackage{mathdots}
\usepackage{multirow}
\usepackage{centernot}
\usepackage{tensor}
\usepackage{fancyhdr}
\usepackage{lastpage}
\usepackage{enumitem}
\usepackage{amsthm}
\usepackage{mathabx}
\usepackage{algpseudocode}
\usepackage{algorithm}


\title{Sprawozdanie 2 Obliczenia Naukowe}
\author{Piotr Zapała}
\date{Novemaber 2022}
\begin{document}
\maketitle

\tableofcontents
\newpage
\begin{center}
    \section{Zadanie 1}
    \subsection{Opis problemu}

    \subsection{Rozwiązanie}
    \begin{flushleft}

    \end{flushleft}

     \begin{flushleft}

    \end{flushleft}

     \begin{flushleft}

    \end{flushleft}

    \subsection{Wyniki}

    \subsection{Wnioski}

    \section{Zadanie 2}
    \subsection{Opis problemu}

    \subsection{Rozwiązanie}

    \subsection{Wyniki}

    \subsection{Wnioski}

    \section{Zadanie 3}
    \subsection{Opis problemu}

    \subsection{Rozwiązanie}

    \subsection{Wyniki}

    \subsection{Wnioski}

    \section{Zadanie 4}
    \subsection{Opis problemu}

    \subsection{Rozwiązanie}

    \subsection{Wyniki}

    \subsection{Wnioski}

    \section{Zadanie 5}
    \subsection{Opis problemu}
 
    \subsection{Rozwiązanie}

    \subsection{Wyniki}

    \subsection{Wnioski}

    \section{Zadanie 6}
    \subsection{Opis problemu}

    \subsection{Rozwiązanie}
   
    \subsection{Wnioski}

    \section{Zadanie 7}
    \subsection{Opis problemu}

    \subsection{Rozwiązanie}

    \subsection{Wyniki}
    
    \subsection{Wnioski}

    \end{center}

\end{document}