\documentclass{article}
\usepackage[utf8]{inputenc}
\usepackage{fancyhdr}
\usepackage{polski}
\usepackage{mathtools}
\usepackage{ulem}
\usepackage[margin = 1cm]{geometry}
\usepackage{hhline}
\usepackage{array}
\usepackage{booktabs}
\usepackage{tabularx}
\usepackage{colortbl}
\usepackage{longtable}
\usepackage{mathdots}
\usepackage{multirow}
\usepackage{centernot}
\usepackage{tensor}
\usepackage{fancyhdr}
\usepackage{lastpage}
\usepackage{enumitem}
\usepackage{amsthm}
\usepackage{mathabx}
\usepackage{algpseudocode}
\usepackage{algorithm}

\title{Sprawozdanie 2 Obliczenia Naukowe}
\author{Piotr Zapała}
\date{November 2023}
\begin{document}
\maketitle
\graphicspath{ {/Users/piotrzapala/JPG}{/Users/piotrzapala/JPG} }

\tableofcontents
\newpage
\begin{center}
    \section{Zadanie 1}
    \subsection{Opis problemu}
    \large W zadaniu pierwszym jesteśmy proszeni, aby powtórz eksperyment z pierwszej listy laboratoryjnej. Różnica polega na małej modyfikacji danych, należało usuniąć ostatniej 9 z \(x_{4}\) oraz ostatniej 7 z \(x_{5}\).
     W ramach przypomnienia, w zadaniu piątym z pierwszej listy mieliśmy zaimplementować, na cztery różne sposoby, algorytm obliczania iloczynu skalarnego 
     wektorów, a następnie należało obliczyć ten iloczyn dla następującej pary wektorów:
    \[\displaystyle x = [2.718281828, -3.141592654, 1.414213562, 0.577215664, 0.301029995]\]
    \[\displaystyle y = [1486.2497, 878366.9879, -22.37492, 4773714.647, 0.000185049]\]
    \begin{flushleft}
       \vspace*{1cm}
       (a) ``w przód'' \(\textstyle \sum_{i=1}^n x_{i}y_{i}\), tj. algorytm
       \begin{algorithmic}
       \State$S:=0$
       \For{i:=1 \textbf{to} n}
           \State $S:=S+x_{i}*y_{i}$
       \EndFor
       \end{algorithmic}
       \vspace*{1cm}
       (b) ``w tył'' \(\textstyle \sum_{i=n}^1 x_{i}y_{i}\), tj. algorytm
       \begin{algorithmic}
           \State$S:=0$
           \For{i:=n \textbf{downto} 1}
               \State $S:=S+x_{i}*y_{i}$
           \EndFor
           \end{algorithmic}
       \vspace*{1cm}
       (c) dodatnie dodajemy w porządku od największego do najmniejszego, a ujemne w porządku od najmniejszego do największego,
       następnie dodajemy do siebie obliczone sumy częściowe.\newline
       (d) przeciwnie do metody (c). 
    \end{flushleft}
    \subsection{Rozwiązanie}
    \large Rozwiązanie polega na zaimplementowaniu podanych algorytmów, a następnie obliczeniu iloczynu skalarnego dla każdego z osobna.
    \subsection{Wyniki}

    \begin{table}[h!]
    \centering
    \begin{tabular}{||c c c||} 
    \hline
    \textbf{type} & \texttt{Float32} & \texttt{Float64} \\ [0.5ex]
    \hline\hline
    \textbf{example1} & -0.4999443 & 1.0251881368296672e-10 \\ 
    \textbf{example2} & -0.4543457 & -1.5643308870494366e-10 \\ 
    \textbf{example3} & -0.5 & 0.0 \\ 
    \textbf{example4} & -0.5 & 0.0 \\ 
    \hline
    \end{tabular}
    \caption{iloczyn skalarny.}
    \label{table:1}
    \end{table}
     
    \begin{table}[h!]
    \centering
    \begin{tabular}{||c c c||} 
    \hline
    \textbf{type} & \texttt{Float32} & \texttt{Float64} \\ [0.5ex]
    \hline\hline
    \textbf{example1} & -0.4999443 & -0.004296342739891585 \\ 
    \textbf{example2} & -0.4543457 & -0.004296342998713953 \\ 
    \textbf{example3} & -0.5 & -0.004296342842280865 \\ 
    \textbf{example4} & -0.5 & -0.004296342842280865 \\ 
    \hline
    \end{tabular}
    \caption{iloczyn skalarny z modyfikacją.}
    \label{table:2}
    \end{table}
    \newpage
    \subsection{Wnioski}
    \large Analizując powyższe wyniki można łatwo zauważyć, że nie zmieniły się dla \texttt{Float32} względem zadania 5 z listy 1. Jest to spowodowane niewielką precyzją arytmetyki. 
     Porównując z arytmetyką \texttt{Float64} można dojść do wniosku, że pomimo nie wprowadzenia zmian w algorytmach struktura \texttt{Float32} dała zgoła inne wyniki niż \texttt{Float64}.
     Tutaj niewielka zmiana rzutowała na wynik iloczynu skalarnego bardzo mocno. 
     Do rozwiązywania takich zadań należy używać maksymalnej precyzji. 
     Zadanie jest źle uwarunkowane, małe decyzje w danych wejściowych powodują większe zaburzenia. 
     Wgłębiając się bardziej w strukturę problemu błędu możemy dojść do wniosku, że modyfikacja danych wejściowych o najmniejszą cyfrę znaczącą w tym wypadku rzędu \(10^{-10}\) jest inaczej zapamiętywana przez różne standardy.
     Właśnie ze względu na różnicę liczby bitów przeznaczonych na zapamiętanie mantysy. Mamy sytuację, w której \texttt{Float32} nie widzi różnicy bo wszystkie zmiany zachodzą poza zakresem mantysy.
     Zaś \texttt{Float64} ma większą precyzję, większą liczbę bitów na zapamiętanie mantysy co za tym idzie obejmuje swoim obszarem mantysy zmiany, co widać w wynikach.

    \section{Zadanie 2}
    \subsection{Opis problemu}
    \large W zadaniu drugim jesteśmy proszeni o narysowanie wykresu funkcji \(f(x) = e^x*ln(1 + e^{-x})\) w co najmniej dwóch programach.
     Następnie należy policzyć granicę \(\lim_{x \to \infty} e^x*\ln(1 + e^{-x})\), a otrzymany wynik skonfrontować z naszymi wykresami.
    \subsection{Rozwiązanie}
    \large Do narysowania wykresów posłużyłem się \texttt{MATLAB'em} oraz \texttt{GeoGebrą}, granice obliczyłem przy pomocy \texttt{Wolfram Alpha}.
    \subsection{Wyniki}
    \(\lim_{x \to \infty} e^x*\ln(1 + e^{-x}) = 1\)
    \includegraphics[width=\textwidth]{wykres1.png}
    \includegraphics[width=\textwidth]{wykres2.png}
    \subsection{Wnioski}
    \large Przyglądając się powyższym wykresom, możemy dostrzec, iż dla wartości przekraczających 30 nasze wykresy prezentują niespodziewane perturbacje.
     Zgodnie z wyznaczoną granicą oczekiwalibyśmy, że od pewnego miejsca wartości zaczną zbiegać do 1. Natomiast w naszym przypadku, następuje nagły spadek do 0,
     poprzedzony chwilową oscylacją naszych wartości wokół 1. Problem najpewniej wynika ze zbyt małej wartości wykładnika, gdyż następuje wtedy dzielenie jedynki przez bardzo duży składnik znajdujący się w mianowniku.

    \section{Zadanie 3}
    \subsection{Opis problemu}
    \large Zadanie trzecie polega na rozwiązaniu układu równań liniowych \textbf{Ax}=\textbf{b}, dla danej macierzy współczynników  \(\textbf{A} \in R^{n \times n}\) i wektora prawych stron  \(\textbf{b} \in R^n\).
     Do tego celu należy użyć następujących algorytmów: eliminacji Gaussa \textbf{x}=\textbf{A/b} oraz x = \textbf{\(A^{-1}\)}b (x=inv(A)*b).
     Eksperymenty wykonujemy dla dwóch typów macierzy, Hilberta \textbf{\(H_{n}\)} dla \(n>1\) w moim przypadku \((n = 2,4,6\ldots,36)\) oraz macierzy 
     losowej \textbf{\(R_{n}\)}, dla \(n = 5, 10, 20\) z rosnącym wskaźnikiem uwarunkowania \(c = 1,10,10^3,10^7,10^{12},10^{16}\). 
     Następnie policzyć błędy względne \(\frac{\|x - \tilde{x}\|}{\|x\|}\), gdzie \(x = (1,\ldots,1)^T\).
     Przykładowa macierz Hilberta \(3 \times 3\). 
     \[\left[\begin{matrix}    
        1 & \frac{1}{2} & \frac{1}{3} \\[6pt]
        \frac{1}{2} & \frac{1}{3} & \frac{1}{4} \\[6pt]
        \frac{1}{3} & \frac{1}{4} & \frac{1}{5}
     \end{matrix}\right]\]
    \subsection{Rozwiązanie}
    \large Podczas rozwiązywania zadania należało skorzystać z funkcji \textbf{A=hilb(n)}, która zwraca macierz Hilberta o zadanym \textbf{n} oraz 
     \textbf{A=matcond(n,c)} zwracjącą losową macierz z danym wskaźnikiem uwarunkowania. Algorytm eliminacji Gaussa jest częścią języka \texttt{Julia}, 
     a jego wywołanie jest równoważne następującej operacji \textbf{x}=\textbf{A/b}. Aby rozwiązać układ równań drugim sposobem używamy funkcji \textbf{inv(x)}.
    \subsection{Wyniki}
    \begin{table}[h!]
    \centering
    \begin{tabular}{||c c c c c||} 
    \hline
    \textbf{size} & \textbf{rank} & \textbf{cond} & \textbf{gaussian error} & \textbf{invert error} \\ [0.5ex]
    \hline\hline
    2 & 2 & 19.281470067903967 & 5.661048867003676e-16 & 1.1240151438116956e-15 \\
    4 & 4 & 15513.738738929662 & 3.587613383388527e-13 & 2.4396188501254767e-13 \\
    6 & 6 & 1.4951058641931808e7 & 2.6132877364524203e-10 & 2.556358881528184e-10 \\
    8 & 8 & 1.5257576052786306e10 & 3.7551147709171007e-7 & 4.029793620039031e-7 \\
    10 & 10 & 1.602489743907797e13 & 0.00020726510768035548 & 0.0002485713513251291 \\
    12 & 11 & 1.6360718665566702e16 & 0.04676893017350922 & 0.17615857301849733 \\
    14 & 11 & 2.4325396487256118e17 & 3.0194351169450253 & 3.4642402117856057 \\
    16 & 12 & 6.019326309924319e17 & 4.586582236304648 & 5.860603956215107 \\
    18 & 12 & 2.2287877143812815e18 & 4.480862554024397 & 5.339000330241911 \\
    20 & 13 & 1.0843011915344271e18 & 12.655226165432483 & 36.458539959836386 \\
    22 & 13 & 9.63324344620851e18 & 50.98332587273586 & 41.1754108459008 \\
    24 & 13 & 7.765483100170821e18 & 17.042652561190312 & 42.752046249600056 \\
    26 & 14 & 4.872032287682753e18 & 12.366008990359907 & 18.455425454123958 \\
    28 & 14 & 5.969866145859385e18 & 91.81593379606892 & 80.87151296965483 \\
    30 & 14 & 3.8417252904919854e18 & 22.757911310802164 & 31.231681742390187 \\
    32 & 14 & 7.226010642658811e18 & 91.18468026004645 & 97.01384267591234 \\
    34 & 14 & 7.877864065765033e18 & 16.318780330167282 & 18.52148018667523 \\
    36 & 15 & 1.1984500558524019e19 & 31.4344690401547 & 44.67185207319411 \\
    \hline
    \end{tabular}
    \caption{Hilbert}
    \label{table:4}
    \end{table}

    \begin{table}[h!]
    \centering
    \begin{tabular}{||c c c c c||} 
    \hline
    \textbf{size} & \textbf{rank} & \textbf{cond} & \textbf{gaussian error} & \textbf{invert error} \\ [0.5ex]
    \hline\hline
    5 & 5 & 1.0000000000000007 & 1.7901808365247238e-16 & 1.4043333874306804e-16 \\
    5 & 5 & 9.999999999999995 & 4.864753555590494e-16 & 5.438959822042073e-16 \\
    5 & 5 & 1000.0000000000095 & 4.0296660774695415e-15 & 3.7224769761911e-15 \\
    5 & 5 & 9.999999988980627e6 & 8.733377064403182e-11 & 4.8241189717112e-11 \\
    5 & 5 & 9.999816990527867e11 & 1.1133476564496114e-5 & 1.849016181079592e-5 \\ 
    5 & 4 & 1.5951433239650136e16 & 0.21622357302232126 & 0.2711884395501898 \\
    10 & 10 & 1.0000000000000007 & 3.3857251850959236e-16 & 3.1597501217190306e-16 \\ 
    10 & 10 & 9.999999999999998 & 2.3551386880256624e-16 & 2.2752801345137457e-16 \\
    10 & 10 & 1000.0000000000496 & 3.321272584248899e-14 & 3.348876922871696e-14 \\
    10 & 10 & 1.0000000001124866e7 & 2.706753549009248e-10 & 2.3291364309905156e-10 \\ 
    10 & 10 & 1.0001159201077572e12 & 1.3351437040014653e-5 & 2.0643407181520504e-5 \\
    10 & 9 & 6.379387724567673e15 & 0.059063649540189334 & 0.029978697058601657 \\
    20 & 20 & 1.0000000000000018 & 5.002167713849564e-16 & 5.534436722516086e-16 \\
    20 & 20 & 10.000000000000007 & 7.69985892130607e-16 & 4.1910000110727263e-16 \\
    20 & 20 & 999.9999999999699 & 2.553266764881752e-14 & 2.6184651813276296e-14 \\
    20 & 20 & 1.0000000002053203e7 & 2.2513786794432707e-10 & 2.0531536109299356e-10 \\ 
    20 & 20 & 1.0000836964768766e12 & 1.9390807211897133e-5 & 1.5496514406250566e-5 \\
    20 & 19 & 5.81750942337733e15 & 0.04332188790837964 & 0.17286837675040306 \\
    \hline
    \end{tabular}
    \caption{Random}
    \label{table:5}
    \end{table}
    \newpage
    \subsection{Wnioski}
    \large Pierwszym nasuwającym się wnioskiem jest fakt, że nawet dla małych rozmiarów macierze Hilberta, 
     mają bardzo duże wskaźniki uwarunkowania. Kolejną rzeczą jest to, iż w przypadku macierzy Hilberta widzimy znaczącą 
     różnice w wartościach błędów względnych, względem macierzy generowanej losowo.
     W przypadku macierzy generowanej losowo, nie jesteśmy w stanie stwierdzić wyższości jednego algorytmu nad drugim.
     Dla Hilberta sytuacja wygląda zgoła inaczej, gdyż algorytm eliminacji Gaussa daje wyniki bliższe temu rzeczywistemu.
     Zatem nasuwającym się wnioskiem jest to, że rozwiązywanie układu równań dla macierzy Hilberta jest zadaniem źle uwarunkowanym.

    \section{Zadanie 4}
    \subsection{Opis problemu}
    \large W zadaniu czwartym należy przy użyciu funkcji \texttt{roots} obliczyć 20 miejsc zerowym wielomianu P.
   \[ 
    \begin{array}{c}
        P(x) = x^{20} - 210x^{19} + 20615x^{18} -1256850x^{17} + 53327946x^{16} \\
         - 1672280820x^{15} + 40171771630x^{14} - 756111184500x^{13} \\
         + 11310276995381x^{12} - 135585182899530x^{11} \\
         + 1307535010540395x^{10} - 10142299865511450x^{9} \\
         + 63030812099294896x^{8} - 311333643161390640x^{7} \\
         + 1206647803780373360x^{6} - 3599979517947607200x^{5} \\
         + 8037811822645051776x^{4} - 12870931245150988800x^{3} \\ 
         + 13803759753640704000x^{2} - 8752948036761600000x \\
         + 2432902008176640000 \\
    \end{array}
    \]
    \large P jest postacią naturalną wielomianu Wilkinsona p.
    \[ 
    \begin{array}{c}
        p(x) = (x - 20)(x - 19)(x - 18)(x - 17)(x - 16) \\
                (x - 15)(x - 14)(x - 13)(x - 12)(x - 11) \\
                     (x - 10)(x - 9)(x - 8)(x - 7)(x - 6) \\
                       (x - 5)(x - 4)(x - 3)(x - 2)(x - 1) \\ 
    \end{array}
    \]
    \large Następnie należy sprawdzić wyznaczone pierwiastki \(z_{k}\), 
     obliczając \(|P(z_{k})|\), \(|p(z_k)|\) i \(|z_{k} - k|\), gdzie \(k \in [1,\ldots,20]\). 
     Drugim poleceniem jest to, aby powtórz eksperyment, ale z modyfikacją współczynnika \(-210\) na \(-210-2^{-23}\).
    \subsection{Rozwiązanie}
    \large Do rozwiązania zadania posłużyłem się trzema funkcjami z pakietu \texttt{Polynomials}. 
     Pierwszą jest \textbf{polynomials(coefficients)} tworząca równanie w postaci ogólnej z podanych współczynników,
     \textbf{fromroots(1:20)} zwracająca wielomian na podstawie jego miejsc zerowych oraz
    \textbf{roots(polynomial)}, która oblicza pierwiastki podanego wielomianu. 
    \subsection{Wyniki}

    \begin{table}[h!]
    \centering
    \begin{tabular}{||c c c||} 
    \hline
    \textbf{\(P(z_{k})\)} & \textbf{\(p(z_{k})\)} & \textbf{\(|z_{k} - k|\)}  \\ [0.5ex]
    \hline\hline
    1603.9573920179469 & 2627.95739201793 & 1.6653345369377348e-14 \\
    7032.126244053075 & 9351.873755953971 & 8.602007994795713e-13 \\
    363020.53660426877 & 280076.5365949661 & 3.3646685437815904e-10 \\ 
    4.1297007568024816e6 & 3.867556758775393e6 & 3.0104239989725556e-8 \\ 
    3.075695036659063e7 & 3.0116950254288312e7 & 8.773618791479976e-7 \\
    1.4257935012621844e8 & 1.4125224900969258e8 & 1.3036540314814715e-5 \\ 
    5.24632388586803e8 & 5.2217372324751204e8 & 0.00011769796309923919 \\
    1.7358866242071867e9 & 1.7316932035664785e9 & 0.0007083981118194416 \\
    5.1002248626897955e9 & 5.093502113334706e9 & 0.003039097772564503 \\
    1.166611991635824e10 & 1.1655899158883463e10 & 0.009425364817383652 \\ 
    3.127208130563727e10 & 3.125699514215297e10 & 0.02299817354506395 \\
    6.1418141120587135e10 & 6.1397123899342445e10 & 0.040566592069646745 \\ 
    1.5300549815452048e11 & 1.5297585245239474e11 & 0.05950452401079254 \\ 
    3.0985557081238544e11 & 3.0981676989206445e11 & 0.06480991092200128 \\
    5.010255098229966e11 & 5.009726725103758e11 & 0.05398301833971253 \\
    9.394123675655612e11 & 9.393457079007137e11 & 0.03609096183973115 \\
    2.3971581253245e12 & 2.39707235924346e12 & 0.0165393639009217 \\
    5.670957056381105e12 & 5.67084966955888e12 & 0.005602369314754441 \\ 
    7.741019369389632e12 & 7.740885904434366e12 & 0.0011451039164107613 \\
    1.2178388919649346e13 & 1.2178225081520236e13 & 0.00011119342792653697 \\
    \hline
    \end{tabular}
    \caption{\(-210\)}
    \label{table:6}
    \end{table}

    \begin{table}[h!]
    \centering
    \begin{tabular}{||c c c||} 
    \hline
    \textbf{\(P(z_{k})\)} & \textbf{\(p(z_{k})\)} & \textbf{\(|z_{k} - k|\)}  \\ [0.5ex]
    \hline\hline
    10130.71275321466 & 10130.712753214342 & 7.749356711883593e-14 \\
    56308.28771988236 & 72692.28772015421 & 8.29603052920902e-12 \\
    807092.163108728 & 631988.1628406073 & 8.906004822506475e-10 \\
    8.084860855411538e6 & 7.298429054000744e6 & 5.614410936161107e-8 \\ 
    3.879198510577002e7 & 3.4091787996578604e7 & 1.0868296920207854e-6 \\ 
    1.4732257627258718e8 & 6.416108188442689e7 & 5.193150526494605e-6 \\
    4.1411759028405327e8 & 9.853716087511169e8 & 0.0002283049351108346 \\
    5.02184047848722e8 & 1.676272204696169e10 & 0.006980931819212444 \\
    1.2260299344962182e9 & 1.3459810989298729e11 & 0.08227581314042176 \\ 
    1.711412152637466e9 & 1.483751062525329e12 & 0.6505270479399019 \\
    1.711412152637466e9 & 1.483751062525329e12 & 1.1105047852610384 \\
    1.922343284948516e10 & 3.2961283180123582e13 & 1.665389446835571 \\
    1.922343284948516e10 & 3.2961283180123582e13 & 2.0460081179778995 \\
    2.78104055609644e11 & 9.547548928471629e14 & 2.518885830105916 \\
    2.78104055609644e11 & 9.547548928471629e14 & 2.712916579182928 \\
    1.0504983293359739e12 & 2.7422169135681444e16 & 2.9060058491454366 \\ 
    1.0504983293359739e12 & 2.7422169135681444e16 & 2.8254814771679904 \\
    6.884949792026641e12 & 4.252502050102871e17 & 2.454019284846941 \\
    6.884949792026641e12 & 4.252502050102871e17 & 2.004325976483215 \\
    9.72908249032009e11 & 1.3743613755543311e18 & 0.8469082068719089 \\
    \hline
    \end{tabular}
    \caption{\(-210-2^{-23}\)}
    \label{table:7}
    \end{table}

    \subsection{Wnioski}
    \large Spoglądając na otrzymane wyniki widzimy, iż zarówno wyliczając wartość z postaci normalnej jak i iloczynowej, nie byliśmy w stanie uzyskać spodziewanego wyniku.
     Jednoznacznie wskazuje nam to, iż mamy do czynienia z bardzo "patologicznym" przypadkiem, ponieważ miejsca zerowe nie zerują naszych funkcji.
     Wartości naszych wielomianów przyjmują wręcz niebotyczne wartości, a fakt iż, nawet tak pozornie mała modyfikacja współczynnika z \(-210\) na \(-210-2^{-23}\), ma znaczący wpływ na wyniki końcowe.
     Sprawia, że nasuwa się oczywisty wniosek, mianowice to zadanie z pewnością jest źle uwarunkowane.   

    \section{Zadanie 5}
    \subsection{Opis problemu}
    \large W zadaniu piątym mamy przedstawiony model wzrostu populacji, 
    \[ p_{n} := p_{n} + rp_{n}(1 - p{n}), n = 0,1,\ldots, \]
    r jest pewną stałą, \(r(1 - p_{n})\) jest czynnikiem wzrostu populacji, a \(p_{0}\) jest wielkością 
    populacji stanowiącą procent maksymalnej wielkości populacji dla danego stanu środowiska.
    Dla danych \(p_{0} = 0.01\) i \(r = 3\) należy w arytmetyce \texttt{Float32} 
    wykonać 40 iteracji wyrażenia, a następnie ponownie wykonać 40 iteracji z tą różnicą, iż po 10 iteracjach 
    odrzucamy cyfry po trzecim miejscu po przecinku i dopiero po obcięciu wyniku prowadzimy dalsze obliczenia. 
    Następnie należy wykonać po 40 iteracji wyrażenia w arytmetyce \texttt{Float32} i \texttt{Float64}, a następnie porównać wyniki.

    \subsection{Rozwiązanie}
    \large Rozwiązanie polega na obliczaniu kolejnych wartości \(p_{n+1}\) w pętli.
    \subsection{Wyniki}
    \begin{table}[h!]
    \centering
    \begin{tabular}{||c c||} 
    \hline
    \textbf{type} & \textbf{result}\\ [0.5ex]
    \hline\hline
    \texttt{\(iterations = 40\)} & 0.25860548  \\ 
    \texttt{\(iterations = 10 + 30\)} & 1.093568 \\
    \hline
    \end{tabular}
    \caption{przykład a}
    \label{table:7}
    \end{table}

    \begin{table}[h!]
    \centering
    \begin{tabular}{||c c||} 
    \hline
    \textbf{type} & \textbf{result} \\ [0.5ex]
    \hline\hline
    \textbf{Float32} & 0.25860548  \\ 
    \textbf{Float64} & 0.011611238029748606 \\
    \hline
    \end{tabular}
    \caption{przykład b}
    \label{table:8}
    \end{table}

    \subsection{Wnioski}
    \large Wyniki zaprezentowane powyżej, jednoznacznie ukazują wpływ zastosowanej przez nas redukcji cyfr po przecinku.
     W naszym przypadku nastąpiło obcięcie do trzech cyfr znaczących, pomimo tego różnica w uzyskanych wynikach jest duża.
     Podobna sytuacja ma miejsce w przypadku wykorzystania innej arytmetyki, tylko zamiast zmniejszać precyzje, zwiększamy ją.
     Dzięki temu prowadząc obliczenia jesteśmy mniej narażeni na utratę cyfr, które mogą mieć realny wpływ na wynik końcowy.

    \section{Zadanie 6}
    \subsection{Opis problemu}
    \large W zadaniu szóstym ponownie mamy do rozważenia równanie rekurencyjne
    \[x_{n+1} := x^{2}_{n} + c, n = 0,1,\ldots,\]
    gdzie c jest pewną daną stałą. Należy przeprowadzić szereg eksperymentów, polegających 
    na wykonywaniu po 40 iteracji powyższego wyrażenia w arytmetyce \texttt{Float64} dla różnych wartości \(c\) oraz \(x_{0}\).
    \begin{enumerate}
        \item c = -2, \(x_0 = 1\)
        \item c = -2, \(x_0 = 2\)
        \item c = -2, \(x_0 = 1.99999999999999\)
        \item c = -1, \(x_0 = 1\)
        \item c = -1, \(x_0 = -1\)
        \item c = -1, \(x_0 = 0.75\)
        \item c = -1, \(x_0 = 0.25\)
    \end{enumerate}
    \newpage
    \subsection{Rozwiązanie}
        \begin{table}[h]
        \centering
        \begin{tabular}{|c|c|c|c|}
        \hline
        $n$ & $x_0=1$ & $x_0=2$ & $x_0 = 1.99999999999999$ \\
        \hline
        1 & -1.0 & 2.0 & 1.99999999999996 \\
        2 & -1.0 & 2.0 & 1.9999999999998401 \\
        3 & -1.0 & 2.0 & 1.9999999999993605 \\
        4 & -1.0 & 2.0 & 1.999999999997442 \\
        5 & -1.0 & 2.0 & 1.9999999999897682 \\
        6 & -1.0 & 2.0 & 1.9999999999590727 \\
        7 & -1.0 & 2.0 & 1.999999999836291 \\
        8 & -1.0 & 2.0 & 1.9999999993451638 \\
        9 & -1.0 & 2.0 & 1.9999999973806553 \\
        10 & -1.0 & 2.0 & 1.999999989522621 \\
        11 & -1.0 & 2.0 & 1.9999999580904841 \\
        12 & -1.0 & 2.0 & 1.9999998323619383 \\
        13 & -1.0 & 2.0 & 1.9999993294477814 \\
        14 & -1.0 & 2.0 & 1.9999973177915749 \\
        15 & -1.0 & 2.0 & 1.9999892711734937 \\
        16 & -1.0 & 2.0 & 1.9999570848090826 \\
        17 & -1.0 & 2.0 & 1.999828341078044 \\
        18 & -1.0 & 2.0 & 1.9993133937789613 \\
        19 & -1.0 & 2.0 & 1.9972540465439481 \\
        20 & -1.0 & 2.0 & 1.9890237264361752 \\
        21 & -1.0 & 2.0 & 1.9562153843260486 \\
        22 & -1.0 & 2.0 & 1.82677862987391 \\
        23 & -1.0 & 2.0 & 1.3371201625639997 \\
        24 & -1.0 & 2.0 & -0.21210967086482313 \\
        25 & -1.0 & 2.0 & -1.9550094875256163 \\
        26 & -1.0 & 2.0 & 1.822062096315173 \\
        27 & -1.0 & 2.0 & 1.319910282828443 \\
        28 & -1.0 & 2.0 & -0.2578368452837396 \\
        29 & -1.0 & 2.0 & -1.9335201612141288 \\
        30 & -1.0 & 2.0 & 1.7385002138215109 \\
        31 & -1.0 & 2.0 & 1.0223829934574389 \\
        32 & -1.0 & 2.0 & -0.9547330146890065 \\
        33 & -1.0 & 2.0 & -1.0884848706628412 \\
        34 & -1.0 & 2.0 & -0.8152006863380978 \\
        35 & -1.0 & 2.0 & -1.3354478409938944 \\
        36 & -1.0 & 2.0 & -0.21657906398474625 \\
        37 & -1.0 & 2.0 & -1.953093509043491 \\
        38 & -1.0 & 2.0 & 1.8145742550678174 \\
        39 & -1.0 & 2.0 & 1.2926797271549244 \\
        40 & -1.0 & 2.0 & -0.3289791230026702 \\
        \hline
    \end{tabular}
    \caption{dla $c=-2$}
    \end{table}
    \newpage

    \begin{table}[h]
    \centering
    \begin{tabular}{|c|c|c|c|c|}
    \hline
    $n$ & $x_0=1$ & $x_0=2$ & $x_0=0.75 $ & $x_0=0.25$ \\
    \hline
    1 & 0.0 & 0.0 & -0.4375 & -0.9375 \\
    2 & -1.0 & -1.0 & -0.80859375 & -0.12109375 \\
    3 & 0.0 & 0.0 & -0.3461761474609375 & -0.9853363037109375 \\
    4 & -1.0 & -1.0 & -0.8801620749291033 & -0.029112368589267135 \\
    5 & 0.0 & 0.0 & -0.2253147218564956 & -0.9991524699951226 \\
    6 & -1.0 & -1.0 & -0.9492332761147301 & -0.0016943417026455965 \\
    7 & 0.0 & 0.0 & -0.0989561875164966 & -0.9999971292061947 \\
    8 & -1.0 & -1.0 & -0.9902076729521999 & -5.741579369278327e-6 \\
    9 & 0.0 & 0.0 & -0.01948876442658909 & -0.9999999999670343 \\
    10 & -1.0 & -1.0 & -0.999620188061125 & -6.593148249578462e-11 \\
    11 & 0.0 & 0.0 & -0.0007594796206411569 & -1.0 \\
    12 & -1.0 & -1.0 & -0.9999994231907058 & 0.0 \\
    13 & 0.0 & 0.0 & -1.1536182557003727e-6 & -1.0 \\
    14 & -1.0 & -1.0 & -0.9999999999986692 & 0.0 \\
    15 & 0.0 & 0.0 & -2.6616486792363503e-12 & -1.0 \\
    16 & -1.0 & -1.0 & -1.0 & 0.0 \\
    17 & 0.0 & 0.0 & 0.0 & -1.0 \\
    18 & -1.0 & -1.0 & -1.0 & 0.0 \\
    19 & 0.0 & 0.0 & 0.0 & -1.0 \\
    20 & -1.0 & -1.0 & -1.0 & 0.0 \\
    21 & 0.0 & 0.0 & 0.0 & -1.0 \\
    22 & -1.0 & -1.0 & -1.0 & 0.0 \\
    23 & 0.0 & 0.0 & 0.0 & -1.0 \\
    24 & -1.0 & -1.0 & -1.0 & 0.0 \\
    25 & 0.0 & 0.0 & 0.0 & -1.0 \\
    26 & -1.0 & -1.0 & -1.0 & 0.0 \\
    27 & 0.0 & 0.0 & 0.0 & -1.0 \\
    28 & -1.0 & -1.0 & -1.0 & 0.0 \\
    29 & 0.0 & 0.0 & 0.0 & -1.0 \\
    30 & -1.0 & -1.0 & -1.0 & 0.0 \\
    31 & 0.0 & 0.0 & 0.0 & -1.0 \\
    32 & -1.0 & -1.0 & -1.0 & 0.0 \\
    33 & 0.0 & 0.0 & 0.0 & -1.0 \\
    34 & -1.0 & -1.0 & -1.0 & 0.0 \\
    35 & 0.0 & 0.0 & 0.0 & -1.0 \\
    36 & -1.0 & -1.0 & -1.0 & 0.0 \\
    37 & 0.0 & 0.0 & 0.0 & -1.0 \\
    38 & -1.0 & -1.0 & -1.0 & 0.0 \\
    39 & 0.0 & 0.0 & 0.0 & -1.0 \\
    40 & -1.0 & -1.0 & -1.0 & 0.0 \\
    \hline
    \end{tabular}
    \caption{dla $c=-1$}
    \end{table}
    \subsection{Wyniki}
    \large Wyznaczam iterację graficzną. Jak to zrobić? \\
    Do przeprowadzenia iteracji trzeba najpierw wyznaczyć wykres funkcji \(y - ax(1 - x)\) oraz dwusieczną (przekątną kwadratu). 
    Następnie należy zaznaczyć punkt \(x_{q}\) na osi \(x\) i przeprowadzić linię pionową wychodzącą z punktu \(x_{0}\),
    a kończącą się w momencie przecięcia wykresu funkcji. Od tego punktu zaczynamy rysować linię poziomą do punktu przecięcia z przekątną,
    a stąd znowu linię pionową do punktu przecięcia z wykresem itd.
    \includegraphics[width=\textwidth]{w1.png}
    \begin{center}
        \(x_{n+1}:=x_{n}^2-1\) dla \(x_{0}=-1\) \\
        Jak widzimy dla \(-1\) widać zacyklenie się procesu.
        Jest one wręcz natychmiastowe. Oznacza to, że jest stabilne.
    \end{center}
    \newpage
    \includegraphics[width=\textwidth]{w2.png}
    \begin{center}
        \(x_{n+1}:=x_{n}^2-1\) dla \(x_{0}=0.25\) \\
        Dla \(0.25\) powoli widać zacyklanie się procesu. Tzn. powolną stabilizację.
    \end{center}
    \newpage
    \includegraphics[width=\textwidth]{w3.png}
    \begin{center}
        \(x_{n+1}:=x_{n}^2-1\) dla \(x_{0}=0.75\) \\
        Jeszcze mocniejsze zacyklanie. Nieskończony cykl.
    \end{center}
    \newpage
    \includegraphics[width=\textwidth]{w4.png}
    \begin{center}
        \(x_{n+1}:=x_{n}^2-1\) dla \(x_{0}=1\) \\
        Jak widzimy dla 1 widać zacyklenie się procesu.
        Jest one wręcz natychmiastowe. Oznacza to, że jest stabilne.
    \end{center}
    \newpage
    \includegraphics[width=\textwidth]{w5.png}
    \begin{center}
        \(x_{n+1}:=x_{n}^2-2\) dla \(x_{0}=1.9999999999999\) \\
        Patrząc na uzyskany wykres widzimy, że proces wyraźnie nie zatrzymuje się.
        Zamiast uporządkowanego zachowania w granicy stopniowo 
        (w kolejnych iteracjach jest mocniej zapełniony) zapełnia on (powoli) całą dostępną przestrzeń.
        Zjawisko to, nazywane mieszaniem, jest wskaźnikiem niestabilnego stanu systemu.
        Spoglądając na dane z tabeli znajdziemy potwierdze- nie naszej graficznej reprezentacji.
    \end{center}
    \newpage
    \includegraphics[width=\textwidth]{w6.png}
    \begin{center}
        \(x_{n+1}:=x_{n}^2-2\) dla \(x_{0}=1\) \\
        Zielone odcinki rysują się do punktów stałych w funkcji.
         Patrząc na tabelę możemy zinterpretować to jako brak zmian w kolejnych wartościach ciągu.
    \end{center}
    \newpage
    \includegraphics[width=\textwidth]{w7.png}
    \begin{center}
        \(x_{n+1}:=x_{n}^2-2\) dla \(x_{0}=2\) \\ 
        Tak samo jak w poprzednim przykładzie.
    \end{center}

    \subsection{Wnioski}

    Otrzymane wyniki dla całkowitych wartości \(c\) i \(x_{0}\) nie zaskakują i są zgodne z intuicją.
    Przy liczbach zmiennoprzecinkowych takich jak np. \(x_{0} = 0.25\) i \(x_{0} = 0.75\) wartości zbiegają do całkowitych przez pewien czas,
    a potem po 40 iteracjach zwracają taką wartość, że jej zachowanie wygląda podobnie jakbyśmy przybrali wartości \(x_{0} = 1\) i \(x_{0} = 2\).
    Przybierając \(x_{0} = 1.99999999999999\) znacząco wpływamy na wyniki po 40 iteracjach.
    Wtedy mamy sytuację, że zamiast oczekiwanych wyników gdzie powinniśmy otrzymywać wyniki oddalone od zera to się do niego przybliża.
    Jest to całkowita rozbieżność w wynikach, które oczekujemy, a wynikami otrzymywanymi.
    Reasumując, gdy dane są wartościami całkowitymi ciągi mają poprawne wartości. 
    W przypadku, gdy dane są ułamkami dla małych \(x\) rekurencja wyznacza poprawne wartości, ale jak widać na wykresach od pewnego momentu te wartości są zaokrąglane do wartości całkowitych.
    Najpierw wartości są całkowite i od pewnego momentu liczby przestają być zaokrąglane do wartości całkowitych. Wyznaczanie kolejnych wyrazów ciągów rekurencyjnych może mieć różną stabilność.
    Zależy to od parametrów, które ustawimy. Całkowicie stabilne to np. \(x_{0} = 1\), stabilizujące się np. \(x_{0} = 0.75\), \(x_{0} = 0.25\), niestabilne np. \(x_{0} \approx 1.99\).
    
    \end{center}

\end{document}