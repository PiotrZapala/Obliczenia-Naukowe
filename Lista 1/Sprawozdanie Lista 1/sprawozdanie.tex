\documentclass{article}
\usepackage[utf8]{inputenc}
\usepackage{fancyhdr}
\usepackage{polski}
\usepackage{mathtools}
\usepackage{ulem}
\usepackage[margin = 1cm]{geometry}
\usepackage{hhline}
\usepackage{array}
\usepackage{booktabs}
\usepackage{tabularx}
\usepackage{colortbl}
\usepackage{longtable}
\usepackage{mathdots}
\usepackage{multirow}
\usepackage{centernot}
\usepackage{tensor}
\usepackage{fancyhdr}
\usepackage{lastpage}
\usepackage{enumitem}
\usepackage{amsthm}
\usepackage{mathabx}
\usepackage{algpseudocode}
\usepackage{algorithm}


\title{Sprawozdanie Obliczenia Naukowe}
\author{Piotr Zapała}
\date{October 2022}
\begin{document}
\maketitle

\tableofcontents
\newpage
\begin{center}
    \section{Zadanie 1}
    \subsection{Opis problemu}
    \large Pierwsze zadanie polega na iteracyjnym wyznaczeniu epsilonów maszynowych
     dla wszystkich typów zmiennopozycyjnych zgodnych ze standardem IEE 754, następnie należy
     porównać z wartościami zwracanymi przez funkcje: \texttt{eps(Float16), eps(Float32), eps(Float64)} 
     oraz z danymi zawartymi w pliku \texttt{float.h}.\vspace*{2mm}

     Następnie musimy wyznaczyć liczbę maszynową \textit{eta}
     dla wszystkich typów zmiennopozycyjnych zgodnych ze standardem IEE 754, następnie należy
     porównać z wartościami zwracanymi przez funkcje: \texttt{nextfloat(Float16(0.0)), nextfloat(Float32(0.0)), nextfloat(Float64(0.0))}.\vspace*{2mm}

     Ostatnią rzeczą o którą jesteśmy proszeni w zadaniu pierwszym, jest iteracyjne wyznaczenie liczby (MAX)
     dla wszystkich typów zmiennopozycyjnych zgodnych ze standardem IEE 754, następnie należy
     porównać z wartościami zwracanymi przez funkcje: \texttt{floatmax(Float16), floatmax(Float32), floatmax(Float64)} 
     oraz z danymi zawartymi w pliku \texttt{float.h} lub danymi z wykładu.\vspace*{2mm}
    \subsection{Rozwiązanie}
    \begin{flushleft}
        \large \textbf{1.machine epsilon}
    \end{flushleft}
     Aby wyznaczyć \texttt{\textit{epsilon maszynowy}} dla danego typu zmiennopozycyjnego, należy posłużyć się funkcjami \textbf{type(x)} oraz \textbf{one(type)}.
     Funkcja \textbf{type(x)} zwraca nam \textbf{x} w danym typie zmiennopozycyjnym, a funkcja \textbf{one(type)} zwraca jedynkę w typie podanym jako argument.
     Początkowo liczba \textbf{epsilon} jest równa \textbf{one(type)}, następnie iteracyjnie jest dzielona przez dwa do momentu, aż \newline
     \textbf{one(type) + epsilon / 2 \(=\) one(type)}.
     \begin{flushleft}
        \large \textbf{2.eta}
    \end{flushleft}
     W celu wyznaczenia liczby \texttt{\textit{eta}} ponownie wykorzystujemy funkcje \textbf{type(x)} oraz \textbf{one(type)}.
     Pierwszym krokiem jest inicjalizacja liczby \texttt{\textit{eta}} jako jedynke w danym typie zmiennopozycyjnym,
     następnie iteracyjnie dzielimy ją przez dwa, do momentu aż \textbf{eta / 2 \(=\) 0}.
     \begin{flushleft}
        \large \textbf{3.max}
    \end{flushleft}
     Do wyznaczenia liczby \texttt{\textit{max}} wykorzystujemy funkcje znane z poprzednich przykładów oraz
     \textbf{prevfloat(x)}, zwracającą największą poprzedzającą \textbf{x} liczbę posiadającą reprezentacje w danej arytmetyce.
     Początkowo \texttt{\textit{max}} wynosi \textbf{prevfloat(one(type))}, następnie z każdą iteracją jej wartość jest podwajana.
     Dzieje się to do momentu, aż wartość zwraca przez funkcję \textbf{isinf(2*max)}, wynosi nieskończoność.
    \subsection{Wyniki}
    \begin{table}[h!]
    \centering
    \begin{tabular}{||c c c c||} 
    \hline
    \textbf{type} & \texttt{Float16} & \texttt{Float32} & \texttt{Float64} \\ [0.5ex]
    \hline\hline
    \textbf{machEps(type)} & 0.000977 & 1.1920929f-7 & 2.220446049250313e-16  \\ 
    \textbf{eps(type)} & 0.000977 & 1.1920929f-7 & 2.220446049250313e-16 \\
    \textbf{float.h} & --||-- & 1.192093e-07 & 2.220446e-16\\
    \hline
    \end{tabular}
    \caption{machine epsilon.}
    \label{table:1}
    \end{table}
    \large Precyzję arytmetki zwyczajowo oznaczamy przez \(\epsilon\), wyraża się ją wzorem \( \displaystyle \epsilon  = \frac{1}{2}*\beta^{1-t} \) , gdzie \(\displaystyle t \in [\frac{1}{\beta}, 1)\).\newpage
    \begin{table}[h!]
    \centering
    \begin{tabular}{||c c c c||} 
    \hline
    \textbf{type} & \texttt{Float16} & \texttt{Float32} & \texttt{Float64} \\ [0.5ex]
    \hline\hline
    \textbf{\(\epsilon\)} & \(2^{-10}\) & \(2^{-23}\) & \(2^{-52}\)  \\
    \textbf{type(\(\epsilon\))} & 0.000977 & 1.1920929f-7 & 2.220446049250313e-16 \\
    \textbf{eps(type)} & 0.000977 & 1.1920929f-7 & 2.220446049250313e-16 \\
    \hline
    \end{tabular}
    \caption{precyzja arytmetyki.}
    \label{table:2}
    \end{table}

    \large Zatem precyzja danej arytmetki jest równa \textbf{eps(type)}.

    \begin{table}[h!]
    \centering
    \begin{tabular}{||c c c c||} 
    \hline
    \textbf{type} & \texttt{Float16} & \texttt{Float32} & \texttt{Float64} \\ [0.5ex]
    \hline\hline
    \textbf{eta(type)} & 6.0e-8 & 1.0f-45 & 5.0e-324  \\ 
    \textbf{nextfloat(type(0))} & 6.0e-8 & 1.0f-45 & 5.0e-324 \\
    \textbf{float.h} & --||-- & 1.175494e-38 & 2.225074e-308\\
    \hline
    \end{tabular}
    \caption{eta.}
    \label{table:2}
    \end{table}
    
    \large \(MIN_{sub}\) jest najmniejszą dla danej arytmetyki liczbą w postaci nieznormalizowanej. \newline
    \(\displaystyle MIN_{sub} = 2^{1-t}*2^{c_{min}}\)
    \(t\) jest liczbą cyfr mantysy, przy czym \(t \in [1,2)\), \newline a \(c_{min} = -2^{d-1} + 2\), gdzie \(d\) oznacza liczbę bitów mantysy.
    \begin{table}[h!]
    \centering
    \begin{tabular}{||c c c c||} 
    \hline
    \textbf{type} & \texttt{Float16} & \texttt{Float32} & \texttt{Float64} \\ [0.5ex]
    \hline\hline
    \textbf{\(c_{min}\)} & -14 & -126 & -1022  \\ 
    \textbf{\(MIN_{sub}\)} & \(2^{-24}\) & \(2^{-149}\) & \(2^{-1074}\) \\
    \textbf{\(type(MIN_{sub})\)} & 6.0e-8 & 1.0e-45 & 5.0e-324 \\
    \textbf{eta(type)} & 6.0e-8 & 1.0e-45 & 5.0e-324 \\
    \hline
    \end{tabular}
    \caption{\(MIN_{sub}\).}
    \label{table:3}
    \end{table}
    
    \large Zatem \(MIN_{sub}\) jest równe liczbie \textbf{eta(type)} w danej arytmetyce. \newline

    \(MIN_{nor}\) jest to najmniejszą liczbą jaką można zapisać w danej arytmetyce w postaci znormalizowanej. \newline
    \(MIN_{nor} = 2^{c_{min}}\), a \(c_{min} = -2^{d-1} + 2\).
    \begin{table}[h!]
    \centering
    \begin{tabular}{||c c c c||} 
    \hline
    \textbf{type} & \texttt{Float16} & \texttt{Float32} & \texttt{Float64} \\ [0.5ex]
    \hline\hline
    \textbf{\(c_{min}\)} & -14 & -126 & -1022  \\ 
    \textbf{\(MIN_{nor}\)} & \(2^{-14}\) & \(2^{-126}\) & \(2^{-1022}\) \\
    \textbf{\(type(MIN_{nor})\)} & 6.104e-5 & 1.1754944e-38 & 2.2250738585072014e-308 \\
    \textbf{floatmin(type)} & 6.104e-5 & 1.1754944e-38 & 2.2250738585072014e-308 \\
    \hline
    \end{tabular}
    \caption{\(MIN_{nor}\).}
    \label{table:3}
    \end{table}

    \large Zatem \(MIN_{nor}\) jest równe \textbf{floatmin(type)} w danej arytmetyce. \newline

    \begin{table}[h!]
    \centering
    \begin{tabular}{||c c c c||} 
    \hline
    \textbf{type} & \texttt{Float16} & \texttt{Float32} & \texttt{Float64} \\ [0.5ex]
    \hline\hline
    \textbf{maxFloat(type)} & 6.55e4 & 3.4028235f38 & 1.7976931348623157e308  \\ 
    \textbf{floatmax(type)} & 6.55e4 & 3.4028235f38 & 1.7976931348623157e308 \\
    \textbf{float.h} & --||-- & 3.402823e+38 & 1.797693e+308\\
    \hline
    \end{tabular}
    \caption{float max}
    \label{table:4}
    \end{table}
    
    \subsection{Wnioski}
    \large Wyniki przedstawione w tabelach, jednoznacznie wskazują na to, że standard IEE 754 ma pewne ogarniczenia.
     Wraz ze wzrostem liczby bitów jakie przeznaczamy na zapis danej liczby, pewne niedogodności arytmetyki są mniej zauważalne.
     Zwiększanie liczby bitów, pozwala nam na zwiększanie zakresu liczb na których możemy operować. 
    \section{Zadanie 2}
    \subsection{Opis problemu}
    \large W zadaniu drugim jesteśmy proszeni o przeprowadzenie serię eksperymentów, mających na celu 
     sprawdzenie, czy dla wszystkich typów zmiennopozycyjnych, obliczenie wyrażenia \textbf{3(4/3-1)-1} daje \texttt{\textit{epsilon maszynowy}}.
    \subsection{Rozwiązanie}
    \large W celu rozwiązania zadania drugiego należy posłużyć się funkcjami \textbf{one(type)} oraz \textbf{type(x)}.\newline
    \subsection{Wyniki}
     \begin{table}[h!]
     \centering
     \begin{tabular}{||c c c c||} 
     \hline
     \textbf{type} & \texttt{Float16} & \texttt{Float32} & \texttt{Float64} \\ [0.5ex]
     \hline\hline
     \textbf{eps(type)} & 0.000977 & 1.1920929f-7 & 2.220446049250313e-16  \\ 
     \textbf{kahanEps(type)} & -0.000977 & 1.1920929f-7 & -2.220446049250313e-16 \\
     \hline
     \end{tabular}
     \caption{kahan eps.}
     \label{table:5}
     \end{table}
    
    \subsection{Wnioski}
    \large Przez ogarniczoną dokładność arytmetyki zmiennopozycyjnej, jesteśmy narażeni na błędy nawet w bardzo trywialnych przykładach.
    \section{Zadanie 3}
    \subsection{Opis problemu}
    \large W zadaniu trzecim musimy sprawdzić, czy liczby zmiennopozycyjne w arytmetyce \texttt{Float64} są równo
     rozmieszczone z następującym krokiem \(\delta = 2^{-54}\) w przedziałach: [1,2], [\(\frac{1}{2}\),1] oraz [2,4].
    \subsection{Rozwiązanie}
     \large Korzystamy ze wskazówki i używamy funkcji \textbf{bitstring}, dzięki niej jesteśmy w stanie wypisać reprezentacje binarne liczb.
      Sprawdzamy po 10 pierwszych liczb z danego przedziału, gdyż sprawdzenie całości mogłoby okazać się kłopotliwe.
    \subsection{Wyniki}
    \begin{table}[h!]
    \centering
    \begin{tabular}{||c||} 
    \hline
    \textbf{Pierwsze 10 liczb w przedziale [2, 4]} \\ [0.5ex]
    \hline\hline
    0011111111110000000000000000000000000000000000000000000000000000 \\
    0011111111110000000000000000000000000000000000000000000000000001 \\
    0011111111110000000000000000000000000000000000000000000000000010 \\
    0011111111110000000000000000000000000000000000000000000000000010 \\
    0011111111110000000000000000000000000000000000000000000000000010 \\
    0011111111110000000000000000000000000000000000000000000000000011 \\
    0011111111110000000000000000000000000000000000000000000000000100 \\
    0011111111110000000000000000000000000000000000000000000000000100 \\
    0011111111110000000000000000000000000000000000000000000000000100 \\
    0011111111110000000000000000000000000000000000000000000000000101 \\ 
    \hline
    \end{tabular}
    \caption{przedział [2, 4].}
    \label{table:5}
    \end{table}
    \begin{table}[h!]
    \centering
    \begin{tabular}{||c||} 
    \hline
    \textbf{Pierwsze 10 liczb w przedziale [1, 2]} \\ [0.5ex]
    \hline\hline
    0011111111110000000000000000000000000000000000000000000000000001 \\
    0011111111110000000000000000000000000000000000000000000000000010 \\
    0011111111110000000000000000000000000000000000000000000000000011 \\
    0011111111110000000000000000000000000000000000000000000000000100 \\
    0011111111110000000000000000000000000000000000000000000000000101 \\
    0011111111110000000000000000000000000000000000000000000000000110 \\
    0011111111110000000000000000000000000000000000000000000000000111 \\
    0011111111110000000000000000000000000000000000000000000000001000 \\
    0011111111110000000000000000000000000000000000000000000000001001 \\
    0011111111110000000000000000000000000000000000000000000000001010 \\
    \hline
    \end{tabular}
    \caption{przedział [1, 2]}
    \label{table:5}
    \end{table}

    \begin{table}[h!]
    \centering
    \begin{tabular}{||c||} 
    \hline
    \textbf{Pierwsze 10 liczb w przedziale [1/2, 1]} \\ [0.5ex]
    \hline\hline
    0011111111110000000000000000000000000000000000000000000000000010 \\
    0011111111110000000000000000000000000000000000000000000000000100 \\
    0011111111110000000000000000000000000000000000000000000000000110 \\
    0011111111110000000000000000000000000000000000000000000000001000 \\
    0011111111110000000000000000000000000000000000000000000000001010 \\
    0011111111110000000000000000000000000000000000000000000000001100 \\
    0011111111110000000000000000000000000000000000000000000000001110 \\
    0011111111110000000000000000000000000000000000000000000000010000 \\
    0011111111110000000000000000000000000000000000000000000000010010 \\ 
    0011111111110000000000000000000000000000000000000000000000010100 \\
    \hline
    \end{tabular}
    \caption{przedział [1/2, 1]}
    \label{table:5}
    \end{table}

    \large Możemy zauważyć, iż gęstość rozmieszenia liczb w przedziale [2, 4] jest dwukrotnie większa niż w przypadku przedziału [1, 2].

    \subsection{Wnioski}
    \large Z powyższego przykładu widzimy, iż to jaka odległość dzieli kolejne liczby (jaka jest precyzja arytmetyki)
     mające reprezentację w danej arytmetyce, zależy od przedziału w którym znajdują się te liczby.
    \section{Zadanie 4}
    \subsection{Opis problemu}
    \large W pierwszym podpunkcie zadania czwartego jesteśmy proszeni o to, aby w arytmetyce \texttt{Float64} \newline
     zgodnej ze standardem IEE 754 znaleźć taką liczbę \textit{x} w przedziale \(1<x<2\), że \(\textit{x}*(\frac{1}{\textit{x}}) \neq 1\). \newline
     Natomiast podpunkt drugi polega na wyznaczeniu najmniejszej takiej liczby.
    \subsection{Rozwiązanie}
    \large W tym zadaniu należy się posłużyć funkcjami \textbf{one(type)}, \textbf{zero(type)} oraz funkcją \textbf{nextfloat(x)}.
     Funkcja \textbf{nextfloat(x)}, zwaraca najmniejszą liczbę, większą od \textbf{x}, która posiada reprezentację w danej arytmetyce zmiennopozycyjnej.
     W przykładzie pierwszym zaczynamy od jedynki, następnie wyznaczamy iteracyjnie następną liczbę posiadającą reprezentację w naszej arytmetyce.
     Czynność powtarzamy do momentu, aż nie znajdziemy liczby która nie spełnia danej równość \(\textit{x}*(\frac{1}{\textit{x}}) = 1\).
     Przykład drugi jest analogiczny, z taką różnicą, że tym razem zaczynamy od zera. \newpage
    \subsection{Wyniki}
    \begin{flushleft}
     \large Wyniki prezentują się następująco: \newline
     1. \(x = 1.000000057228997\) \newline
     2. \(x = 1.0e-323\)
    \end{flushleft}
    \subsection{Wnioski}
    \large Analizując podane przykłady możemy wywnioskować, iż IEE 754 ma ograniczoną
     dokładność i nawet bardzo proste operacje mogą dawać fałszywe wyniki.
    \section{Zadanie 5}
    \subsection{Opis problemu}
    \large W zadaniu piątym jesteśmy proszeni o zaimplemetowanie, na cztery różne sposoby, algorytm obliczania iloczynu skalarnego 
     wektorów, a następnie mamy obliczyć ten iloczyn dla następującej pary wektorów:
     \[\displaystyle x = [2.718281828, -3.141592654, 1.414213562, 0.5772156649, 0.3010299957]\]
     \[\displaystyle y = [1486.2497, 878366.9879, -22.37492, 4773714.647, 0.000185049]\]
     \begin{flushleft}
        \vspace*{1cm}
        (a) ``w przód'' \(\textstyle \sum_{i=1}^n x_{i}y_{i}\), tj. algorytm
        \begin{algorithmic}
        \State$S:=0$
        \For{i:=1 \textbf{to} n}
            \State $S:=S+x_{i}*y_{i}$
        \EndFor
        \end{algorithmic}
        \vspace*{1cm}
        (b) ``w tył'' \(\textstyle \sum_{i=n}^1 x_{i}y_{i}\), tj. algorytm
        \begin{algorithmic}
            \State$S:=0$
            \For{i:=n \textbf{downto} 1}
                \State $S:=S+x_{i}*y_{i}$
            \EndFor
            \end{algorithmic}
        \vspace*{1cm}
        (c) dodatnie dodajemy w porządku od największego do najmniejszego, a ujemne w porządku od najmniejszego do największego,
        następnie dodajemy do siebie obliczone sumy częściowe.\newline
        (d) przeciwnie do metody (c). \newpage
     \end{flushleft}
    \subsection{Rozwiązanie}
    \large Rozwiązanie polega na zaimplementowaniu podanych algorytmów, a następnie obliczeniu iloczynu skalarnego dla każdego z osobna.
    \subsection{Wyniki}

    \begin{table}[h!]
    \centering
    \begin{tabular}{||c c c||} 
    \hline
    \textbf{type} & \texttt{Float32} & \texttt{Float64} \\ [0.5ex]
    \hline\hline
    \textbf{example1} & -0.49999443f0 & -5.0134667188046684e-5 \\ 
    \textbf{example2} & -0.4543457f0 & -5.013464760850184e-5 \\ 
    \textbf{example3} & -0.5f0 & -5.013449117541313e-5 \\ 
    \textbf{example5} & -0.5f0 & -5.013495683670044e-5 \\ 
    \hline
    \end{tabular}
    \caption{iloczyn skalarny.}
    \label{table:5}
    \end{table}

    \subsection{Wnioski}
    \large Przyglądając się wynikom, nasuwają się dwa wnioski, pierwszym jest fakt, iż na wynik wpływa typ arytmetyki, którą się posługujemy.
     Drugą rzeczą jest to, iż na końcowy wynik ma również wpływ kolejność wykonywania działań.
    \section{Zadanie 6}
    \subsection{Opis problemu}
    \large W zadaniu szóstym jesteśmy proszeni o to, aby w arytmetyce \texttt{Float64} obliczyć wartości następujących funkcji:
     \[\displaystyle f(x) = \sqrt{x^2 + 1} - 1\]
     \[\displaystyle g(x) = \frac{x^2}{\sqrt{x^2 + 1} + 1}\]
     dla kolejnych wartości argumentu \(x = 8^{-1},8^{-2},8^{-3}\),\ldots.
     Następnie należy odpowiedzieć, które wyniki są wiarygodne, bo pomimo iż \(f=g\), komputer daje nam różne rozwiązania.
    \subsection{Rozwiązanie}
    \large Rozwiązanie polega na zaimplementowaniu podanych równań, a następnie obliczaniu ich wartości dla kolejnych argumentów \(x = 8^{-1},8^{-2},8^{-3}\),\ldots.
    \subsection{Wyniki}
    \begin{table}[h!]
        \centering
        \begin{tabular}{||c c c||} 
        \hline
        \textbf{x} & \textbf{f(x)} & \textbf{g(x)}  \\ [0.5ex]
        \hline\hline
         \(8^{-1}\) & 0.0077822185373186414 & 0.0077822185373187065 \\ 
         \(8^{-2}\) & 0.00012206286282867573  & 0.00012206286282875901 \\
         \(8^{-3}\) & 1.9073468138230965e-6 & 1.907346813826566e-6 \\ 
         \(8^{-4}\) & 2.9802321943606103e-8  & 2.9802321943606116e-8 \\ 
         \(8^{-5}\) & 4.656612873077393e-10  & 4.6566128719931904e-10 \\ 
         \(8^{-6}\) & 7.275957614183426e-12  & 7.275957614156956e-12 \\ 
         \(8^{-7}\) & 1.1368683772161603e-13 &  1.1368683772160957e-13 \\ 
         \(8^{-8}\) & 1.7763568394002505e-15 & 1.7763568394002489e-15 \\ 
         \(8^{-9}\) & 0.0 & 2.7755575615628914e-17 \\ 
         \(8^{-10}\) & 0.0 & 4.336808689942018e-19 \\ 
         \(8^{-20}\) & 0.0 & 3.76158192263132e-37 \\ 
         \(8^{-30}\) & 0.0 & 3.2626522339992623e-55 \\ 
         \(8^{-40}\) & 0.0 & 2.8298997121333476e-73 \\ 
         \(8^{-50}\) & 0.0 & 2.4545467326488633e-91 \\ 
         \(8^{-60}\) & 0.0 & 2.1289799200040754e-109 \\ 
         \(8^{-70}\) & 0.0 & 1.8465957235571472e-127 \\ 
         \(8^{-80}\) & 0.0 & 1.6016664761464807e-145 \\ 
         \(8^{-90}\) & 0.0 & 1.3892242184281734e-163 \\ 
         \(8^{-100}\) & 0.0 & 1.204959932551442e-181 \\ 
        \hline
        \end{tabular}
        \caption{porównanie wartości funkcji f i g dla \(x = 8^{-1},8^{-2},8^{-3}\),\ldots.}
        \label{table:5}
        \end{table}
        Pomimo równości \(f\) i \(g\), widzimy znaczącą różnicę w otrzymywanych wynikach.
        W tym przypadku jest to spowodowane odejmowaniem we wzorze funkcji \(f\).
        Dla bardzo małych wartości \(x\), mamy \(\textstyle \sqrt{x^2 + 1} \approx 1\), powoduje to utratę cyfr znaczących z wyniku pierwisatka.
    \subsection{Wnioski}
    \large W przypadkach w których może wystąpić odjemowanie bliskich sobie liczb, należy się zastanowić, 
     czy nie jesteśmy w stanie wykonać przekształcenia algebraicznego. Gdyż może to pozwolić na prowadzenie bardziej dokładnych obliczeń. 
    \section{Zadanie 7}
    \subsection{Opis problemu}
    \large Korzystając ze wzoru na przybliżoną wartość pochodnej \(f(x)\) w punkcie \(x\),
     \[\displaystyle f'(x) \approx\ \tilde{f'}(x_{0}) = \frac{f(x_{0} + h) - f(x_{0})}{h}\]
     w arytmetyce \texttt{Float64} należy obliczyć przybliżoną wartość pochodnej funkcji \(f(x) = \sin x + \cos 3x\) w punkcie \(x_{0}=1\) 
     oraz błędów \(|f'(x_{0}) - \tilde{f'}(x_{0})|\) dla \(h = 2^{-n}\) \((n = 0,1,2,\ldots,54)\). \newline
     Następnie należy wyjaśnić dlaczego od pewnego momentu zmniejszanie wartości \(h\) 
     nie poprawia przybliżenia wartości pochodnej, sprawdzić jak zachowują się wartości \(1 + h\) 
     oraz porównać obliczone przybliżenia pochodnej z jej dokładną wartością.
     Pochodna funkcji \(f\) ma następującą postać \(f'(x) = \cos x - 3\sin 3x\).
    \subsection{Rozwiązanie}
    \large Rozwiązanie polega na zaimplementowaniu podanych równań, a następnie obliczaniu ich wartości dla kolejnych \(h\) oraz \(1 + h\).
    \subsection{Wyniki}
     \large Dokładna wartość pochodnej w punkcie wynosi \(0.11694228168853815\)
     \newpage
     \begin{table}[h!]
     \centering
     \begin{tabular}{||c c c||} 
     \hline
     \textbf{h} & \textbf{derivative(f, h, x0)} & \textbf{relativeError(f, g, h, x0)}  \\ [0.5ex]
     \hline\hline
     1.0 & 2.0179892252685967 & 1.9010469435800585 \\ 
     0.5 & 1.8704413979316472 & 1.753499116243109 \\
     0.25 & 1.1077870952342974 & 0.9908448135457593 \\ 
     0.125 & 0.6232412792975817 & 0.5062989976090435 \\ 
     0.0625 & 0.3704000662035192 & 0.253457784514981 \\ 
     0.03125 & 0.24344307439754687 & 0.1265007927090087 \\ 
     0.015625 & 0.18009756330732785 & 0.0631552816187897 \\ 
     0.0078125 & 0.1484913953710958 & 0.03154911368255764 \\ 
     0.00390625 & 0.1327091142805159 & 0.015766832591977753 \\ 
     0.001953125 &  0.1248236929407085 & 0.007881411252170345 \\ 
     0.0009765625 & 0.12088247681106168 & 0.0039401951225235265 \\ 
     0.000244140625 &  0.11792723373901026 & 0.0009849520504721099 \\ 
     0.0001220703125 & 0.11743474961076572 & 0.0004924679222275685 \\ 
     6.103515625e-5 & 0.11718851362093119 & 0.0002462319323930373 \\ 
     3.0517578125e-5 & 0.11706539714577957 & 0.00012311545724141837 \\ 
     1.52587890625e-5 & 0.11700383928837255 & 6.155759983439424e-5\\ 
     7.62939453125e-6 & 0.11697306045971345 & 3.077877117529937e-5 \\ 
     3.814697265625e-6 & 0.11695767106721178 & 1.5389378673624776e-5 \\ 
     1.9073486328125e-6 & 0.11694997636368498 & 7.694675146829866e-6 \\ 
     9.5367431640625e-7 &  0.11694612901192158 & 3.8473233834324105e-6 \\ 
     4.76837158203125e-7 & 0.1169442052487284 & 1.9235601902423127e-6 \\ 
     2.384185791015625e-7 & 0.11694324295967817 & 9.612711400208696e-7 \\ 
     1.1920928955078125e-7 & 0.11694276239722967 & 4.807086915192826e-7 \\ 
     5.960464477539063e-8 & 0.11694252118468285 & 2.394961446938737e-7 \\ 
     2.9802322387695312e-8 & 0.116942398250103 & 1.1656156484463054e-7 \\ 
     1.4901161193847656e-8 & 0.11694233864545822 & 5.6956920069239914e-8 \\ 
     7.450580596923828e-9 &  0.11694231629371643 & 3.460517827846843e-8 \\ 
     3.725290298461914e-9 & 0.11694228649139404 & 4.802855890773117e-9 \\ 
     1.862645149230957e-9 & 0.11694222688674927 & 5.480178888461751e-8 \\ 
     9.313225746154785e-10 & 0.11694216728210449 & 1.1440643366000813e-7 \\ 
     4.656612873077393e-10 & 0.11694216728210449 & 1.1440643366000813e-7 \\ 
     2.3283064365386963e-10 & 0.11694192886352539 & 3.5282501276157063e-7 \\ 
     1.1641532182693481e-10 & 0.11694145202636719 & 8.296621709646956e-7 \\ 
     5.820766091346741e-11 & 0.11694145202636719 & 8.296621709646956e-7 \\ 
     2.9103830456733704e-11 & 0.11693954467773438 & 2.7370108037771956e-6 \\ 
     1.4551915228366852e-11 & 0.116943359375 & 1.0776864618478044e-6 \\ 
     7.275957614183426e-12 & 0.1169281005859375 & 1.4181102600652196e-5 \\ 
     3.637978807091713e-12 &  0.116943359375 & 1.0776864618478044e-6 \\ 
     1.8189894035458565e-12 & 0.11688232421875 & 5.9957469788152196e-5 \\ 
     9.094947017729282e-13 &  0.1168212890625 & 0.000120992626038152 \\ 
     4.547473508864641e-13 &  0.116943359375 & 1.0776864618478044e-6 \\ 
     2.2737367544323206e-13 & 0.11669921875 & 0.0002430629385381522 \\ 
     1.1368683772161603e-13 & 0.1162109375 & 0.0007313441885381522 \\ 
     5.684341886080802e-14 & 0.1171875 & 0.0002452183114618478 \\ 
     2.842170943040401e-14 &  0.11328125 & 0.003661031688538152 \\ 
     1.4210854715202004e-14 & 0.109375 & 0.007567281688538152 \\ 
     7.105427357601002e-15 & 0.109375 & 0.007567281688538152 \\ 
     3.552713678800501e-15 & 0.09375 & 0.023192281688538152 \\ 
     1.7763568394002505e-15 & 0.125 & 0.008057718311461848 \\ 
     8.881784197001252e-16 & 0.0 & 0.11694228168853815 \\ 
     4.440892098500626e-16 & 0.0 & 0.11694228168853815 \\ 
     2.220446049250313e-16 & -0.5 & 0.6169422816885382 \\ 
     1.1102230246251565e-16 & 0.0 & 0.11694228168853815 \\ 
     5.551115123125783e-17 & 0.0 & 0.11694228168853815 \\ 
     \hline
     \end{tabular}
     \caption{porównanie wyników dla h.}
     \label{table:6}
     \end{table}
     \begin{table}[h!]
     \centering
     \begin{tabular}{||c c c||} 
     \hline
     \textbf{1 + h} & \textbf{derivative(f, 1 + h, x0)} & \textbf{relativeError(f, g, 1 + h, x0)} \\ [0.5ex]
     \hline\hline
     2.0 & -0.3107443710161304 & 0.42768665270466855 \\ 
     1.5 & 0.7290859824876875 & 0.6121437007991494 \\
     1.25 & 1.4556808426956374 & 1.3387385610070992 \\ 
     1.125 & 1.7730038158791617 & 1.6560615341906235 \\ 
     1.0625 & 1.90633124732842 & 1.7893889656398818 \\ 
     1.03125 & 1.9650746526601908 & 1.8481323709716526 \\ 
     1.015625 & 1.992284872504413 & 1.875342590815875 \\ 
     1.0078125 & 2.0053282295936543 & 1.8883859479051162 \\ 
     1.00390625 & 2.011706884941135 & 1.8947646032525967 \\ 
     1.001953125 & 2.0148601393872734 & 1.8979178576987352 \\ 
     1.0009765625 & 2.016427708987274 & 1.8994854272987358 \\ 
     1.000244140625 & 2.0175994143067255 & 1.9006571326181874 \\ 
     1.0001220703125 & 2.0177943671554126 & 1.9008520854668745 \\ 
     1.00006103515625 & 2.017891808055301 & 1.9009495263667628 \\ 
     1.000030517578125 & 2.0179405196229427 & 1.9009982379344046 \\ 
     1.0000152587890625 & 2.01796487318604 & 1.9010225914975019 \\ 
     1.0000076293945312 & 2.017977049412388 & 1.90103476772385 \\ 
     1.0000038146972656 & 2.0179831373867603 & 1.9010408556982221 \\ 
     1.0000019073486328 & 2.0179861813392455 & 1.9010438996507073 \\ 
     1.0000009536743164 & 2.0179877033068125 & 1.9010454216182744 \\ 
     1.0000004768371582 & 2.017988464288428 & 1.9010461825998899 \\ 
     1.000000238418579 & 2.0179888447786927 & 1.9010465630901545 \\ 
     1.0000001192092896 & 2.0179890350236898 & 1.9010467533351516 \\ 
     1.0000000596046448 & 2.0179891301461548 & 1.9010468484576166 \\ 
     1.0000000298023224 & 2.0179891777073786 & 1.9010468960188405 \\ 
     1.0000000149011612 & 2.0179892014879885 & 1.9010469197994504 \\ 
     1.0000000074505806 & 2.0179892133782924 & 1.9010469316897542 \\ 
     1.0000000037252903 & 2.0179892193234443 & 1.9010469376349062 \\ 
     1.0000000018626451 & 2.0179892222960207 & 1.9010469406074826 \\ 
     1.0000000009313226 & 2.0179892237823087 & 1.9010469420937706 \\ 
     1.0000000004656613 & 2.017989224525453 & 1.9010469428369148 \\ 
     1.0000000002328306 & 2.0179892248970246 & 1.9010469432084864 \\ 
     1.0000000001164153 & 2.0179892250828106 & 1.9010469433942725 \\ 
     1.0000000000582077 & 2.0179892251757034 & 1.9010469434871653 \\ 
     1.0000000000291038 & 2.0179892252221503 & 1.9010469435336121 \\ 
     1.000000000014552 & 2.0179892252453735 & 1.9010469435568353 \\ 
     1.000000000007276 & 2.017989225256985 & 1.901046943568447 \\ 
     1.000000000003638 & 2.017989225262791 & 1.901046943574253 \\ 
     1.000000000001819 & 2.0179892252656937 & 1.9010469435771555 \\ 
     1.0000000000009095 & 2.017989225267145 & 1.9010469435786068 \\ 
     1.0000000000004547 & 2.017989225267871 & 1.901046943579333 \\ 
     1.0000000000002274 & 2.017989225268234 & 1.9010469435796957 \\ 
     1.0000000000001137 & 2.017989225268415 & 1.901046943579877 \\ 
     1.0000000000000568 & 2.017989225268506 & 1.901046943579968 \\ 
     1.0000000000000284 & 2.0179892252685514 & 1.9010469435800132 \\ 
     1.0000000000000142 & 2.017989225268574 & 1.901046943580036\\ 
     1.000000000000007 & 2.017989225268585 & 1.901046943580047 \\ 
     1.0000000000000036 & 2.017989225268591 & 1.9010469435800528 \\ 
     1.0000000000000018 & 2.017989225268594 & 1.9010469435800559 \\ 
     1.0000000000000009 & 2.0179892252685954 & 1.9010469435800572 \\ 
     1.0000000000000004 & 2.017989225268596 & 1.9010469435800577 \\ 
     1.0000000000000002 & 2.0179892252685963 & 1.901046943580058 \\ 
     1.0 & 2.0179892252685967 & 1.9010469435800585 \\ 
     1.0 & 2.0179892252685967 & 1.9010469435800585 \\ 
     \hline
     \end{tabular}
     \caption{porównanie wyników dla 1 + h.}
     \label{table:7}
     \end{table}

    \large Najlepsze przybliżenie pochodnej w punkcie \(x_{0} = 1\) otrzymujemy do \(h = 2^{-28}\).
     Zatem zmniejszanie wartości \(h\) od pewnego wcale nie poprawia jakości naszych obliczeń, wręcz przeciwnie.
     Jest to spowodowane odejmowaniem dwóch bliskich sobie liczb.
    \subsection{Wnioski}
    \large Prawadząc tego typu obliczenia, powinniśmy się wystrzegać wartości bliskich zera.


    \end{center}

\end{document}