\documentclass{article}
\usepackage[utf8]{inputenc}
\usepackage{fancyhdr}
\usepackage{polski}
\usepackage{mathtools}
\usepackage{ulem}
\usepackage[margin = 1cm]{geometry}
\usepackage{hhline}
\usepackage{array}
\usepackage{booktabs}
\usepackage{tabularx}
\usepackage{colortbl}
\usepackage{longtable}
\usepackage{mathdots}
\usepackage{multirow}
\usepackage{centernot}
\usepackage{tensor}
\usepackage{fancyhdr}
\usepackage{lastpage}
\usepackage{enumitem}
\usepackage{amsthm}
\usepackage{mathabx}
\usepackage{algpseudocode}
\usepackage{algorithm}
\usepackage{pgfplots}


\title{Sprawozdanie Obliczenia Naukowe}
\author{Piotr Zapała}
\date{November 2023}
\begin{document}
\maketitle

\tableofcontents
\newpage
\begin{center}
    \section{Zadanie 1}
    \subsection{Opis problemu}
    \large W zadaniu pierwszym należy napisać funkcję rozwiązującą równanie \(f(x) = 0\) metodą bisekcji. \newline
     \texttt{function mbisekcji(f::Function, a::Float64, b::Float64, delta::Float64, epsilon::Float64)} 
     \begin{flushleft}
        Dla następujących danych wejściowych: \newline
        \texttt{f} - funkcja \(f(x)\) zadana jako anonimowa funkcja, \newline
        \texttt{a, b} - końce przedziału początkowego, \newline
        \texttt{delta, epsilon} - dokładność obliczeń, \newline
        \newline
        Nasza funkcja powinna zwracać następujące parametry: \newline
        \texttt{r} - przybliżenie pierwiastka równania \(f(x) = 0\), \newline
        \texttt{v} - wartość \(f(r)\), \newline
        \texttt{it} - liczba wykonanych iteracji, \newline
        \texttt{err} - sygnalizacja błędu (\textbf{0} - brak błędu, \textbf{1} - funkcja nie zmienia znaku w przedziale \([a, b]\)), \newline
     \end{flushleft}
     Metoda bisekcji, czyli inaczej metoda połowienia przedziału, jest konsekwencją własności Darboux dla funkcji ciągłych.
     Jeśli f jest funkcją ciągłą w przedziale \([a, b]\) i jeśli \(f(a)f(b) < 0\), a więc f zmienia znak w \([a, b]\),
     to ta funkcja musi mieć zero w \((a, b)\). Zatem jeśli \(f(a)f(b) < 0\), to obliczamy \(c = \frac{1}{2}(a + b)\)
     i sprawdzamy, czy \(f(a)f(c) < 0\). Jeśli tak, to f ma zero w \([a, c]\); wtedy pod \(b\) podstawiamy \(c\).
     W przeciwnym razie jest \(f(c)f(b) < 0\); wtedy pod \(a\) podstawiamy \(c\). W obu przypadkach nowy przedział \([a, b]\), 
     dwa razy krótszy od poprzedniego, zawiera zero funkcji f, zatem postępowanie można powtórzyć. Jeśli \(f(a)f(c) = 0\), to \(f(c) = 0\) i zero zostało znalezione.
     Jednakże z powodu błędów zaokrągleń, sytuacja gdzie \(f(c) = 0\), jest mało prawdopodobna i nie powinna stanowić kryterium zakończenia obliczeń.
     Należy więc dopuścić pewną tolerancję wyników, przykładowo \(|f(c)| < 10^{-5}\), gdy precyzja arytmetyki wynosi \(\epsilon = 2^{-24}\).

    \subsection{Rozwiązanie}
    \large Aby rozwiązać zadanie pierwsze należało zaimplementować w języku \texttt{Julia} następujący algorytm. 
    \begin{flushleft}
        \begin{algorithm}
            \caption{Algorytm Bisekcji}\label{alg:bisekcji}
            \begin{algorithmic}
                \Function{mbisekcji}{$f, a, b, \delta, \epsilon$}
                \State$it \gets 0$
                \State$u \gets f(a)$
                \State$v \gets f(b)$
                \State$e \gets (b - a)$
                \If{$sgn(u)==sgn(v)$}
                \Return{$(0, 0 , it, 1)$}
                \EndIf
                \While{$true$}
                    \State$it \gets it + 1$
                    \State$e \gets e/2$
                    \State$c \gets a + e$
                    \State$w \gets f(c)$
                    \If{$|e| < \delta$ or $|w| < \epsilon$}
                    \Return{$(c, w , it, 0)$}
                    \EndIf
                    \If{$sgn(u) \ne sgn(v)$}
                        \State$b \gets c$
                        \State$v \gets w$
                    \Else
                        \State$a \gets c$
                        \State$u \gets w$
                    \EndIf
                \EndWhile
                \EndFunction
            \end{algorithmic}
        \end{algorithm}
    \end{flushleft}

    \section{Zadanie 2}
    \subsection{Opis problemu}
    \large W zadaniu drugim należy napisać funkcję rozwiązującą równanie \(f(x) = 0\) metodą stycznych. \newline
    \texttt{function mstycznych(f::Function, pf::Function, x0::Float64, delta::Float64, epsilon::Float64, maxit::Int)} \newline
    \begin{flushleft}
       Dla następujących danych wejściowych: \newline
       \texttt{f, pf} - funkcją \(f(x)\) oraz pochodną \(f'(x)\) zadane jako anonimowe funkcje, \newline
       \texttt{x0} - przybliżenie początkowe, \newline
       \texttt{delta, epsilon} - dokładność obliczeń, \newline
       \texttt{maxit} - maksymalna dopuszczalna liczba iteracji, \newline
       \newline
       Nasza funkcja powinna zwracać następujące parametry: \newline
       \texttt{r} - przybliżenie pierwiastka równania \(f(x) = 0\), \newline
       \texttt{v} - wartość \(f(r)\), \newline
       \texttt{it} - liczba wykonanych iteracji, \newline
       \texttt{err} - sygnalizacja błędu (\textbf{0} - metoda zbieżna, \textbf{1} - nie osiągnięto wymaganej dokładności w \texttt{maxit} iteracji, \textbf{2} - pochodna bliska zeru), \newline
    \end{flushleft}
    Metoda stycznych, czyli inaczej metoda Newtona, na ogół szybsza od metody bisekcji i siecznych, ponieważ jej zbieżność jest kwadratowa.
    Gdy tylko przybliżenia tworzone metodą Newtona są dostatecznie bliskie pierwiastka, staje się szybko zbieżna i zaledwie pare kolejnych iteracji daje już maksymalną dokładność.
    Niech \(r\) będzie takim zerem, a \(x\) jego przybliżeniem, Jeśli \(f''\) istnieje, to na mocy twierdzenia \textbf{Taylora}.
    \[0 = f(r) = f(x + h) = f(x) + hf'(x) + O(h^2)\] \newline
    gdzie \(h = r - x\). Jeśli \(h\) jest małe, to możemy pominąć składnik \(O(h^2)\) i rozwiązać otrzymane równanie względem \(h\).
    Otrzymujemy \(h = -f(x)/f'(x)\). Zatem jeśli \(x\) jest przybliżeniem \(r\), to \(x - f(x)/f'(x)\) powinno być lepszym przybliżeniem tego zera.
    Skąd metoda Newtona polega na rekurencyjnym stosowaniu wzoru. 
    \[x_{n+1} := x_{n} - \frac{f(x_{n})}{f'(x_{n})}\] 
    \subsection{Rozwiązanie}
    \large Aby rozwiązać zadanie drugie należało zaimplementować w języku \texttt{Julia} następujący algorytm. 
    \begin{flushleft}
        \begin{algorithm}
            \caption{Algorytm Newtona}\label{alg:stycznych}
            \begin{algorithmic}
                \Function{mstycznych}{$f, pf, x0, \delta, \epsilon, maxit$}
                \State$v \gets f(x0)$
                \If{$|v| < \epsilon$}
                \Return{$(x0, v, 0, 0)$}
                \EndIf
                \If{$|pf(x0)| < \epsilon$}
                \Return{$(0, 0, 0, 2)$}
                \EndIf
                \For{$i \gets 1,\, maxit$}
                    \State$x1 \gets x0 - v/pf(x0)$
                    \State$v \gets f(x1)$
                    \If{$|x1 - x0| < \delta$ or $|v| < \epsilon$}
                    \Return{$(x1, v , i, 0)$}
                    \EndIf
                    \State$x0 \gets x1$
                \EndFor
                \EndFunction
            \end{algorithmic}
        \end{algorithm}
    \end{flushleft}

    \section{Zadanie 3}
    \subsection{Opis problemu}
    \large W zadaniu trzecim należy napisać funkcję rozwiązującą równanie \(f(x) = 0\) metodą siecznych. \newline
    \texttt{function msiecznych(f::Function, x0::Float64, x1::Float64, delta::Float64, epsilon::Float64, maxit::Int)} \newline
    \begin{flushleft}
       Dla następujących danych wejściowych: \newline
       \texttt{f} - funkcja \(f(x)\) zadana jako anonimowa funkcja, \newline
       \texttt{x0, x1} - przybliżenia początkowe, \newline
       \texttt{delta, epsilon} - dokładność obliczeń, \newline
       \texttt{maxit} - maksymalna dopuszczalna liczba iteracji, \newline
       \newline
       Nasza funkcja powinna zwracać następujące parametry: \newline
       \texttt{r} - przybliżenie pierwiastka równania \(f(x) = 0\), \newline
       \texttt{v} - wartość \(f(r)\), \newline
       \texttt{it} - liczba wykonanych iteracji, \newline
       \texttt{err} - sygnalizacja błędu (\textbf{0} - metoda zbieżna, \textbf{1} - nie osiągnięto wymaganej dokładności w \texttt{maxit} iteracji), \newline
    \end{flushleft}
    Jedną z wad metody Newtona jest konieczność obliczania wartości pochodnej funkcji \(f\), której zera szukamy.
    W celu umienięcia tego problemu \textbf{Johan Frederik Steffensen} zaproponował następujący wzór
    \[x_{n+1} := x_{n} - \frac{[f(x_{n})]^2}{f(x_{n} + f(x_{n})) - f(x_{n})}\] \newline
    Aby osiągnąć ten sam efekt również można się posłużyć wzorem na iloraz różnicowy.
    \[f'(x_{n}) \approx \frac{f(x_{n}) - f(x_{n-1})}{x_{n} - x_{n-1}}\] \newline
    Powyższa przybliżona równość, bezpośrednio daje nam metodę siecznych opisaną poniższym wzorem.
    \[x_{n+1} := x_{n} - f(x_{n})\frac{x_{n} - x_{n+1}}{f(x_{n}) - f(x_{n-1})} \quad (n \geqslant 1)\]
    \includegraphics[width=100mm]{sieczna.png}
    \subsection{Rozwiązanie}
    \large Aby rozwiązać zadanie trzecie należało zaimplementować w języku \texttt{Julia} następujący algorytm. 
    \begin{flushleft}
        \begin{algorithm}
            \caption{Algorytm siecznych}\label{alg:siecznych}
            \begin{algorithmic}
                \Function{msiecznych}{$f, x0, x1, \delta, \epsilon, maxit$}
                \State $fa \gets f(x0)$
                \State $fb \gets f(x1)$
                \For{$i \gets 1,\, maxit$}
                    \If{$|fa| > |fb|$}
                    \State$x0 \leftrightarrow x1$
                    \State$fa \leftrightarrow fb$
                    \EndIf
                    \State$s \gets (x1 - x0)/(fb - fa)$
                    \State$x1 \gets x0$
                    \State$fb \gets fa$
                    \State$x0 \gets x0 - (fa * s)$
                    \State$fa \gets f(x0)$
                    \If{$|fa| < \epsilon$ or $|x1 - x0| < \delta$}
                    \Return{$(x0, fa, i, 0)$}
                    \EndIf
                \EndFor \newline
                \Return{$(0, 0, 0, 1)$}
                \EndFunction
            \end{algorithmic}
        \end{algorithm}
    \end{flushleft}
    
    \section{Zadanie 4}
    \subsection{Opis problemu}
    \large W zadaniu czwartym, należało wyznaczyć pierwiastki równania \(\sin(x) - (\frac{1}{2}x)^2 = 0\),
     za pomocą metod zaimplementowanych w poprzednich zadaniach.\newline
     \newline
     \begin{flushleft}
        1. metoda bisekcji z przedziałem początkowym \([1.5,2]\) oraz dokładnością obliczeń \(\delta = \frac{1}{2}10^{-5}\), \(\epsilon = \frac{1}{2}10^{-5}\) \newline
        2. metoda Newtona z przybliżeniem początkowym \(x_{0} = 1.5\) oraz dokładnością obliczeń \(\delta = \frac{1}{2}10^{-5}\), \(\epsilon = \frac{1}{2}10^{-5}\) \newline
        3. metoda siecznych z przybliżeniami początkowymi \(x_{0} = 1\), \(x_{1} = 2\) oraz dokładnością \(\delta = \frac{1}{2}10^{-5}\), \(\epsilon = \frac{1}{2}10^{-5}\)
     \end{flushleft}
     Powyższe równanie posiada dwa miejsca zero, \(x=0\) oraz \(x \approx 1.93375\) 
    \subsection{Rozwiązanie}
    \large W celu rozwiązania powyższych przykładów, należało zaimplementować program testujący, który wykorzystuje funkcje z modułu napisanego w zadaniach 1-3.
    \subsection{Wyniki}
    \begin{table}[h!]
        \centering
        \begin{tabular}{||c c c c||} 
        \hline
        \textbf{algorytm} & \textbf{x} & \textbf{f(x)} & \textbf{iteracje} \\ [0.5ex]
        \hline\hline
         bisekcji & 1.9337539672851562  & -2.7027680138402843e-7 & 16 \\
         Newtona & 1.933753779789742 & -2.2423316314856834e-8 & 4 \\
         siecznych & 1.933753644474301 & 1.564525129449379e-7 & 4 \\
        \hline
        \end{tabular}
        \caption{Zadanie 4}
        \label{table:1}
    \end{table} 
    
    \subsection{Wnioski}
    \large Jak widzimy z powyższej tabeli, nasze metody z powodzeniem zwracają poprawne wyniki równania \(\sin(x) - (\frac{1}{2}x)^2 = 0\).
     Przy okazji analizowania tego typu doświadczenia, należy zwrócić uwagę na kwestie liczby iteracji. Wyniki wskazują same za siebie,
     metoda Newtona oraz metoda siecznych są szybciej zbieżne od metody bisekcji. Przy okazji opisu zadania drugiego nadmnieniłem,
     że metoda Newtona jest szybsza od metody bisekcji, gdyż w przypadku tej pierwszej zbieżność jest kwadratowa, a nie liniowa bądź nadliniowa.
     Nasze wyniki faktycznie potwierdzają fakt dotyczący zbieżności użytych metod.
    \section{Zadanie 5}
    \subsection{Opis problemu}
     \large W zadaniu piątym jesteśmy proszeni o to, aby użyć metody bisekcji do wyznaczenia
      punktu przecięcia wykresów następujących funkcji \(y = 3x\) oraz \(y = e^x\). 
      Dodatkowo należało zachować następujące dokładności obliczeń \(\delta = 10^{-4}\) oraz \(\epsilon = 10^{-4}\).
    \subsection{Rozwiązanie}
     \large Aby rozwiązać powyższe zadanie, należało napisać program testujący wykorzystujący funkcję \newline
     \texttt{function mbisekcji(f::Function, a::Float64, b::Float64, delta::Float64, epsilon::Float64)}. \newline
     Nastepnie należało wykonać, przekształcenie powyższych równań do następującej postaci: \newline
     \[e^x = 3x \iff e^x - 3x = 0\] \newline
     Aby wyznaczyć przedział początkowy posłużyłem się programem \texttt{GeoGebra}, z poniższego wykresu wynika, że nasze funkcje przecinają się w dwóch miejscach.
     Pierwsze rozwiązanie zawiera się w przedziale \([0.0, 1.0]\), natomiast drugie w \([1.0, 2.0]\), przybliżone wartości miejsc zerowych to \(x \approx 0.619061\) oraz \(x \approx 1.51213\). \newline
     \newline
     \includegraphics[width=100mm]{3xex.png}
    \subsection{Wyniki}
    \large Poniższa tabela zawiera wyniki przeprowadzonych doświadczeń.
    \begin{table}[h!]
        \centering
        \begin{tabular}{||c c c c||} 
        \hline
        \textbf{przedział} & \textbf{x} & \textbf{f(x)} & \textbf{iteracje} \\ [0.5ex]
        \hline\hline
         \([0.0, 1.0]\) & 0.619140625 & -9.066320343276146e-5 & 9 \\
         \([1.0, 2.0]\) & 1.5120849609375 & -7.618578602741621e-5 & 13 \\
        \hline
        \end{tabular}
        \caption{Zadanie 5}
        \label{table:2}
    \end{table}

    \subsection{Wnioski}
    \large Porównując wartości otrzymane z metody bisekcji oraz wartości obliczone przy pomocy \texttt{Wolfram Alpha}, możemy 
     stwierdzić, że metoda bisekcji daje wiarygone i jak najbardziej zadawalające wyniki dla naszego przypadku. Niemniej jednak, należy
     zwrócić uwagę na dość istotny szczegół, mianowicie jakość wyników końcowych oraz liczba iteracji jest mocno uzależniona od przedziału początkowego.

    \section{Zadanie 6}
    \subsection{Opis problemu}
    \large W zadaniu szóstym, należy znaleźć miejsca zerowe funkcji \(f_{1}(x) = e^{1-x} - 1\) oraz \(f_{2}(x) = xe^{-x}\), za pomocą
     metod bisekcji, Newtona i siecznych. Kolejnym wymaganiem naszego zadania jest, zastosowanie następujących dokładności obliczeń \(\delta = 10^{-5}\), \(\epsilon = 10^{-5}\) oraz 
     dobranie odpowiednich przedziałów i punktów początkowych.    
     Następnie jesteśmy proszeni o sprawdzenie, co się stanie w metodzie Newtona dla \(f_{1}(x) = e^{1-x} - 1\), gdy wybierzemy \(x_{0} \in (1, \infty]\), 
     a dla \(f_{2}(x) = xe^{-x}\) wybierzemy \(x_{0} > 1\). 
    \subsection{Rozwiązanie}
    \large Rozwiązanie polega na zaimplementowaniu programu testującego, który dla \(f_{1}\), \(f_{2}\), danych przedziałów i punktów początkowych, oraz podanej dokładności obliczeń wyznaczy miejsca zerowe funkcji.
     Aby dobrać odpowiednie punkty oraz przedziały początkowe, posłużyłem się programem \texttt{GeoGebra}. \newline
     \newline
     \includegraphics[width=100mm]{ee.png} \newline
     \newline
     Na podstawie wykresu z pewnością możemy stwierdzić, że miejsce zerowe funkcji \(f_{1}\) (wykres zielony) zawiera się w przedziale \([0.0, 2.0]\), 
     natomiast miejsce zerowe \(f_{2}\) (wykres niebieski) zawiera się w przedziale \([-1.0, 1.0]\). Możemy z dużym prawdopodobieństwem założyć, 
     że nasze pierwiastki znajdują się dokładnie w połowie powyższych przedziałów. Aby lepiej zaobserwować zachowanie naszym algorytmów
     należy wybrać nieco inny przedział, gdyż przykładowo dla przedziału \([0.0, 2.0]\) w metodzie bisekcji najprawdopodobniej już pierwsza iteracja zwróci nam poprawny wynik. \newline
     \newline
     Wybrane przedziały oraz punkty początkowe przedstawiam w poniższej tabeli. 

     \begin{table}[h!]
        \centering
        \begin{tabular}{||c c c|} 
        \hline
        \textbf{algorytm} & \textbf{funkcja} & \textbf{parametry} \\ [0.5ex]
        \hline\hline
         bisekcji & \(f_{1}\) & \([0.5, 2.0]\) \\
         bisekcji & \(f_{2}\) &  \([-1.5, 0.5]\) \\
         Newtona & \(f_{1}\) & \(x_{0}=0\) \\
         Newtona & \(f_{2}\) & \(x_{0}=-1\) \\
         siecznych & \(f_{1}\) & \(x_{0}=0.5, x_{1}=2.0\)\\
         siecznych & \(f_{2}\) & \(x_{0}=-1.5, x_{1}=0.5\) \\
        \hline
        \end{tabular}
        \caption{przedziały oraz punkty początkowe}
        \label{table:3}
    \end{table} 
    \subsection{Wyniki}
    \large Wyniki z przeprowadzonych doświadczeń przedstawiam w poniższych tabelach. \newline
    \begin{table}[h!]
        \centering
        \begin{tabular}{||c c c c c||} 
        \hline
        \textbf{algorytm} & \textbf{parametry} & \textbf{x} & \textbf{f(x)} & \textbf{iteracje} \\ [0.5ex]
        \hline\hline
         bisekcji & \([0.5, 2.0]\) & 0.9999923706054688 & 7.629423635080457e-6 & 16 \\
         Newtona & \(x_{0}=0\) & 0.9999984358892101 & 1.5641120130194253e-6 & 4 \\
         siecznych & \(x_{0}=0.5, x_{1}=2.0\) & 1.000000014307199 & -1.4307198870078253e-8 & 6 \\
        \hline
        \end{tabular}
        \caption{funkcja \(f_{1}(x)= e^{1-x} - 1\)}
        \label{table:1}
    \end{table} 

    \begin{table}[h!]
        \centering
        \begin{tabular}{||c c c c c||} 
        \hline
        \textbf{algorytm} & \textbf{parametry} & \textbf{x} & \textbf{f(x)} & \textbf{iteracje} \\ [0.5ex]
        \hline\hline
         bisekcji & \([-1.5, 0.5]\) & 0.0 & 0.0 & 2 \\
         Newtona & \(x_{0}=-1\) & -3.0642493416461764e-7 & -3.0642502806087233e-7 & 5 \\
         siecznych & \(x_{0}=-1.5, x_{1}=0.5\) & 1.7791419742860986e-8 & 1.779141942632637e-8 & 8 \\
        \hline
        \end{tabular}
        \caption{funkcja \(f_{2}(x)= xe^{-x}\)}
        \label{table:1}
    \end{table} 

    \subsection{Wnioski}
    \large Z powyższych wyników, możemy potwierdzić wcześniejsze przypuszczenia dotyczące wartości pierwiastków równań \(f_{1}\) i \(f_{2}\).
     W przypadku \(f_{1}\) jest to \(x=1\), a dla \(f_{2}\) mamy \(x=0\). Najbardziej uwidaczniającą się obserwacją jest to, że obcięcie przedziału \([0.0, 2.0]\) do \([0.5, 2.0]\)
     dla metody bisekcji, poskutkowało zwiększeniem liczby iteracji potrzebnych do wyznaczenia zera funkcji \(f_{1}\).
     Dla funkcji \(f_{2}\) w metodzie bisekcji, zastosowaliśmy przedział \([-1.5, 0.5]\), którego środkiem jest \(0\). Zatem nie powinien dziwić nas fakt, iż
     metoda bisekcji już po dwóch iteracjach zwraca poprawny wynik. Kolejny raz otrzymane wyniki potwierdzają fakt, iż czas potrzebny na wyznaczenie pierwiastków metodą bisekcji jest uzależniony od wyboru przedziału początkowego.
     Jeśli chodzi o resztę metod, to można powiedzieć, że wyniki są zgodne z naszymi oczekiwaniami. 
     Pod warunkiem, że środkiem wybranego przez nas przedziału początkowego nie jest pierwiastek danego równania, należy się spodziewać, że metoda Newtona i siecznych dadzą wynik po mniejszej liczbie iteracji niż metoda połowienia przedziału (bisekcji). Zgodnie z poleceniem, należało jeszcze sprawdzić co się stanie w metodzie Newtona, gdy dla \(f_{1}\) wybierzemy \(x_{0} \in (1, \infty]\), oraz dla \(f_{2}\),
     \(x_{0} > 1\). W moim przypadku wybrałem \(x_{0} = 50\), przy próbie wykonania obliczeń dla funkcji
     \(f_{1}\) otrzymałem komunikat o błędzie, natomiast dla \(f_{2}\) wynikiem jest \(x=50\), wtedy wartość funkcji dla tego argumentu jest równa \(f_{2}(50) = 9.643749239819589e-21\). Generalnie rzecz biorąc wartość otrzymana dla \(f_{2}(50)\) jest bliższa zeru niż wartość \(f_{2}\), którą wyznaczyliśmy prowadząc obliczenia dla punktu początkowego \(x_{0} =-1.0\). W tym konkretnym przypadku, metoda Newtona 
     wskazuje liczbę \(50\) jako pierwiastek równania \(f_{2}\), jednak nie jest to prawda. Wykres \(f_{2}\) nie pozostawia złudzeń i jednoznacznie wskazuje na to co odpowiada, za powyższe nieścisłości. 
     Granica \(\lim_{x \to \infty}f_{2}(x) = 0\), zatem dla \(x_{0} \in (1, \infty]\) funkcja jest zbieżna do zera, a wartości bliskie zera są błędnie interpretowane jako miejsca przecięcia z osią OX. 
     Natomiast dla punktu początkowego \(x_{0}=1\), metoda Newtona zwraca komunikat o błędzie.
     Powodem takiego obrotu spraw jest to, że wartość pochodnej \(f'_{2}(x) = -e^{-x}(x-1)\) 
     w punkcie \(x_{0} = 1\) wynosi \(f'_{2}(1) = 0\), co jest jednoznaczne z tym że styczna leży na osi OX.
    \end{center}
    \begin{thebibliography}{100}
        \bibitem{Kincaid} D. Kincaid, W. Cheney, Analiza numeryczna, Wydawnictwa Naukowo-Techniczne, Warszawa 2006. ISBN 83-204-3078-X
    \end{thebibliography}

\end{document}